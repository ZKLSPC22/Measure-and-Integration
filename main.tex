\documentclass{article}
\usepackage[utf8]{inputenc}
\usepackage{amsmath}
\usepackage[margin=1.8cm]{geometry}
\usepackage{amsfonts}
\usepackage{amssymb}
\usepackage{mathtools}
\usepackage{tikz}
\usepackage{extarrows}
\title{Measure and Integration}
\author{Zhikun Li}
\date{}

\newcommand{\C}{\mathbb{C}}
\newcommand{\R}{\mathbb{R}}
\newcommand{\Z}{\mathbb{Z}}
\newcommand{\E}{\mathbb{E}}
\newcommand{\N}{\mathbb{N}}
\newcommand{\Q}{\mathbb{Q}}
\newcommand{\p}{\mathcal{P}}
\newcommand{\ucov}{\rightrightarrows}
\newcommand{\mdec}{\searrow}
\newcommand{\minc}{\nearrow}

\newcommand{\sumhinf}{\displaystyle\sum_{h=1}^\infty}
\newcommand{\sumiinf}{\displaystyle\sum_{i=1}^\infty}
\newcommand{\sumjinf}{\displaystyle\sum_{j=1}^\infty}
\newcommand{\sumkinf}{\displaystyle\sum_{k=1}^\infty}
\newcommand{\sumlinf}{\displaystyle\sum_{l=1}^\infty}
\newcommand{\summinf}{\displaystyle\sum_{m=1}^\infty}
\newcommand{\sumninf}{\displaystyle\sum_{n=1}^\infty}
\newcommand{\sumpinf}{\displaystyle\sum_{p=1}^\infty}
\newcommand{\sumrinf}{\displaystyle\sum_{r=1}^\infty}
\newcommand{\sumsinf}{\displaystyle\sum_{s=1}^\infty}
\newcommand{\sumtinf}{\displaystyle\sum_{t=1}^\infty}

\newcommand{\infcap}{\displaystyle\bigcap_{n=1}^\infty}
\newcommand{\caphinf}{\displaystyle\bigcap_{h=1}^\infty}
\newcommand{\capiinf}{\displaystyle\bigcap_{i=1}^\infty}
\newcommand{\capjinf}{\displaystyle\bigcap_{j=1}^\infty}
\newcommand{\capkinf}{\displaystyle\bigcap_{k=1}^\infty}
\newcommand{\caplinf}{\displaystyle\bigcap_{l=1}^\infty}
\newcommand{\capminf}{\displaystyle\bigcap_{m=1}^\infty}
\newcommand{\capninf}{\displaystyle\bigcap_{n=1}^\infty}
\newcommand{\cappinf}{\displaystyle\bigcap_{p=1}^\infty}
\newcommand{\caprinf}{\displaystyle\bigcap_{r=1}^\infty}
\newcommand{\capsinf}{\displaystyle\bigcap_{s=1}^\infty}
\newcommand{\captinf}{\displaystyle\bigcap_{t=1}^\infty}

\newcommand{\infcup}{\displaystyle\bigcup_{n=1}^\infty}
\newcommand{\cuphinf}{\displaystyle\bigcup_{h=1}^\infty}
\newcommand{\cupiinf}{\displaystyle\bigcup_{i=1}^\infty}
\newcommand{\cupjinf}{\displaystyle\bigcup_{j=1}^\infty}
\newcommand{\cupkinf}{\displaystyle\bigcup_{k=1}^\infty}
\newcommand{\cuplinf}{\displaystyle\bigcup_{l=1}^\infty}
\newcommand{\cupminf}{\displaystyle\bigcup_{m=1}^\infty}
\newcommand{\cupninf}{\displaystyle\bigcup_{n=1}^\infty}
\newcommand{\cuppinf}{\displaystyle\bigcup_{p=1}^\infty}
\newcommand{\cuprinf}{\displaystyle\bigcup_{r=1}^\infty}
\newcommand{\cupsinf}{\displaystyle\bigcup_{s=1}^\infty}
\newcommand{\cuptinf}{\displaystyle\bigcup_{t=1}^\infty}

\newcommand{\sumin}{\displaystyle\sum_{i=1}^n}
\newcommand{\sumim}{\displaystyle\sum_{i=1}^m}
\newcommand{\sumik}{\displaystyle\sum_{i=1}^k}
\newcommand{\sumjn}{\displaystyle\sum_{j=1}^n}
\newcommand{\sumjm}{\displaystyle\sum_{j=1}^m}
\newcommand{\sumjk}{\displaystyle\sum_{j=1}^k}
\newcommand{\sumkn}{\displaystyle\sum_{k=1}^n}
\newcommand{\sumkm}{\displaystyle\sum_{k=1}^m}
\newcommand{\summn}{\displaystyle\sum_{m=1}^n}
\newcommand{\summk}{\displaystyle\sum_{m=1}^k}
\newcommand{\sumnm}{\displaystyle\sum_{n=1}^m}
\newcommand{\sumnk}{\displaystyle\sum_{n=1}^k}

\newcommand{\cupin}{\displaystyle\bigcup_{i=1}^n}
\newcommand{\cupim}{\displaystyle\bigcup_{i=1}^m}
\newcommand{\cupik}{\displaystyle\bigcup_{i=1}^k}
\newcommand{\cupjn}{\displaystyle\bigcup_{j=1}^n}
\newcommand{\cupjm}{\displaystyle\bigcup_{j=1}^m}
\newcommand{\cupjk}{\displaystyle\bigcup_{j=1}^k}
\newcommand{\cupnk}{\displaystyle\bigcup_{n=1}^k}
\newcommand{\cupnm}{\displaystyle\bigcup_{n=1}^m}
\newcommand{\cupkn}{\displaystyle\bigcup_{k=1}^n}
\newcommand{\cupkm}{\displaystyle\bigcup_{k=1}^m}

\newcommand{\capin}{\displaystyle\bigcap_{i=1}^n}
\newcommand{\capim}{\displaystyle\bigcap_{i=1}^m}
\newcommand{\capik}{\displaystyle\bigcap_{i=1}^k}
\newcommand{\capjn}{\displaystyle\bigcap_{j=1}^n}
\newcommand{\capjm}{\displaystyle\bigcap_{j=1}^m}
\newcommand{\capjk}{\displaystyle\bigcap_{j=1}^k}
\newcommand{\capnk}{\displaystyle\bigcap_{n=1}^k}
\newcommand{\capnm}{\displaystyle\bigcap_{n=1}^m}
\newcommand{\capkn}{\displaystyle\bigcap_{k=1}^n}
\newcommand{\capkm}{\displaystyle\bigcap_{k=1}^m}

\newcommand{\bcsl}{\symbol{92}}
\newcommand{\infprod}{\prod\limits_{n=1}^\infty}
\newcommand{\limiinf}{\displaystyle\lim_{i\to\infty}}
\newcommand{\limninf}{\displaystyle\lim_{n\to\infty}}
\newcommand{\limxinf}{\displaystyle\lim_{x\to\infty}}
\newcommand{\limkinf}{\displaystyle\lim_{k\to\infty}}
\newcommand{\limxx}{\lim_{x\to x_0}}
\newcommand{\difx}{\dfrac{\mbox{d}}{\mbox{d}x}}
\newcommand{\dift}{\dfrac{\mbox{d}}{\mbox{d}t}}
\newcommand{\st}{\mbox{ s.t. }}
\newcommand{\he}{\mbox{ and }}
\newcommand{\theorem}{\textbf{Theorem:}}
\newcommand{\clear}{\mbox{clearly}:}
\newcommand{\trivial}{\mbox{Trivially: }}
\newcommand{\wlg}{\mbox{ WLoG, let }}
\newcommand{\diam}{\mbox{diam}}
\newcommand{\0}{{\bf{0}}}
\newcommand{\1}{{\bf{1}}}
\newcommand{\esssup}{\mbox{ess}\sup\limits}
\newcommand{\essinf}{\mbox{ess}\inf\limits}
\newcommand{\alev}{\mbox{ a.e.}}
\newcommand{\emset}{\emptyset}
\newcommand{\dint}{\displaystyle\int}
\newcommand{\dlim}{\displaystyle\lim}
\newcommand{\intab}{\displaystyle\int_a^b}
\newcommand{\dif}{\mbox{d}}
\newcommand{\incto}{\nearrow}
\newcommand{\decto}{\searrow}

\begin{document}
\maketitle
\section*{Preface}
The following Notes will include all the material covered in ``Measure, Integral and Probability" by Capinski and Kopp as well as ``Real Analysis" by Stein and Shakarchi.
This set of notes aims at being concise, comprehensive, and clear, to act as a complete refresher for revision as well as a dictionary for theorems and proofs.
\tableofcontents
\clearpage
\section{Motivation and Preliminaries}
\subsection{Notation and Basic Set Theory}
\subsubsection{Basic Set Theory}
\null\hfill{See ``Classical Real Analysis Notes"}
\subsubsection{Topological Properties of Sets in $\R$}
\textbf{Definition: Open Set}

$O\subset\R$ is open if for some open intervals $\{I_\alpha\}_{\alpha\in\Lambda}$, $O=\displaystyle\bigcup_{\alpha\in\Lambda}I_\alpha$, where $\Lambda$ is some index set.

Equivalently, $O\subset\R$ is open if for all $x \in O,$ there exists $\delta > 0$ such that $(x - \delta, x + \delta) \subset O.$

Warm-up exercise: show these are indeed equivalent.

\textbf{Definition: Continuity}\\
$f:X\to\R$ is continuous if $\forall$ open sets $O\subset \R$ we have $f^{-1}(O)$ open.\\
In more general metric spaces, this prevents boundary points of $f^{-1}(O)$ from being mapped to an interior point by not allowing the existence of boundary points in $f^{-1}(O)$.\\
\textbf{Notation:} $\lim\sup_nx_n$, $\lim\inf_nx_n$\\
These denote the upper limit and lower limit of $x_n$: if we define the sequences $y_n = \sup\{x_m: m \geq n\}$ and $z_n = \inf \{x_m: m \geq n\}$ then $y_n \xrightarrow{} \limsup{x_n}$ and $z_n \xrightarrow{} \liminf{x_n}.$
\subsection{Limitations of the Riemann Integral}
\textbf{Scope:} The Riemann integral is reliant on limits when dealing with unbounded functions\\
\textbf{Dependence on Intervals:} Ugly domains and functions with very ``scattered" values can't be integrated, such as the indicator function of the rationals.\\
\textbf{The Lack of Completeness:}\\ Integrable functions may converge to an unintegrable function.\\
\textbf{Example:}
$$f_n(x)=\begin{cases}
    4n^2x&0\le x<\dfrac{1}{2n}\\[6pt]
    4n-4n^2x&\dfrac{1}{2n}\le x<\dfrac{1}{n}\\[6pt]
    0&\dfrac{1}{n}\le x\le1
\end{cases}$$
Continuous functions with convergent ``distances" may converge to a discontinuous function.\\
\textbf{Example:}
$$g_n(x)=\begin{cases}
    0&0\le x\le\dfrac{1}{2}\\[6pt]
    n\left(x-\dfrac{1}{2}\right)&\dfrac{1}{2}<x<\dfrac{1}{2}+\dfrac{1}{n}\\
    1&\mbox{otherwise}
\end{cases}$$
The ``distance'' $dint_0^1|g_n(x)-g_m(x)|dx\to0$ as $m,n\to0$\\
\null\hfill{Verify the above examples as an exercise}
\clearpage
\section{Measure}
\subsection{Null Sets}
\textbf{Definition: Length of intervals on $\R$}\\
Clearly, we want to define the length of intervals as the difference between their two ends.
$$\forall a,b\in\R,l([a,b])=l([a,b))=l((a,b])=l((a,b))=b-a$$
$$\implies\forall a\in\R:l(\{a\})=l([a,a])=a-a-0\land l(\emptyset)=l((a,a))=a-a=0$$
It is only natural for finite addition of length to hold.\\
\textbf{Definition: Null Set}
$$A\subset\R,\forall\varepsilon>0,\exists\{I_n\}_{n\in\N^+}\st A\subset\bigcup_{n=1}^\infty I_n\land\sum_{n=1}^\infty l(I_n)<\varepsilon$$
\textbf{Corollary:} Restricting $I_n$ to be all in the form $(a,b)$ or all $[a,b]$ or $[a,b)$ or $(a,b]$ does not change the definition\\
\textbf{Corollary:} Countable sets are null\\
\textbf{Proof:}
$$\forall\{x_n\}_{n\in\N^+},\forall\varepsilon>0,I_n:=\left(x_n-\frac{\varepsilon}{2^{n+2}},x_n+\frac{\varepsilon}{2^{n+2}}\right)\implies\sum_{n=1}^\infty l(I_n)=\frac{\varepsilon}{2}<\varepsilon$$
\textbf{Theorem:} Countable unions of Null sets are Null\\
\textbf{Proof:}\\
Suppose $N=\displaystyle\bigcup_{n=1}^\infty N_n\st\forall N_n$ is Null, then each $N_n$ has cover $\{I_{n,k}\}_{n\in\N^+}\st\displaystyle\sum_{k=1}^\infty l(I_{n,k})<\dfrac{\varepsilon}{2^n}$\\
Clearly, $\displaystyle\bigcup_{n,k\in\N^+}I_{n,k}$ is a cover of $N$ and $\displaystyle\sum_{n,k\in\N^+}l(I_{n,k})=\displaystyle\sum_{n=1}^\infty\displaystyle\sum_{k=1}^\infty l(I_{n,k})=\varepsilon$\\
\textbf{Note:} Uncountable sets can be Null, such as the Cantor set. At each step of the Cantor set's construction, the total length of the covering decreases exponentially. 
\subsection{Outer Measure}
\textbf{Definition: Lebesgue Outer Measure}
$$Z_A:=\left\{\sum_{n=1}^\infty l(I_n)\,\middle\vert\,A\subseteq\bigcup_{n=1}^\infty I_n\right\},\,m^*(A):=\inf Z_A$$
Clearly, $\forall A\subset\R:m^*(A)\geq0$.\\
When all the $\sumninf l(I_n)$ in $Z_A$ diverge to $\infty$, we define $m^*(A)$ as $\infty$.\\
$\forall a\in\R$, define $a+\infty$ as $\infty$, $\infty+\infty:=\infty$, $0\times\infty:=0$\\
Clearly, $Z_A=[r,+\infty]$ or $Z_A=(r,+\infty]$ or $Z_A=[\infty,\infty]=\{\infty\}$, and $A$ is Null$\iff m^*(A)=0$\\
\textbf{Proposition:}
$A\subset B\implies m^*(A)\le m^*(B)$\\
\textbf{Proof:}\\
Clearly, $Z_B\subset Z_A\implies\inf Z_A\le\inf Z_B$\\
\textbf{Theorem:} The outer measure of an interval equals its length$\equiv m^*(I)=l(I)$\\
\textbf{Proof:}\\
$I_1:=I,\forall n\geq2,I_n=[0,0]\implies\displaystyle\bigcup_{n=1}^\infty I_n$ covers $I$, thus $m^*(I)\le l(I)$\\
For bounded $I:\forall\varepsilon>0,\exists\{I_n\}\st\sumninf l(I_n)\le m^*(I)+\dfrac{\varepsilon}{2}\land I\subset\infcup I_n$, let $a_n,b_n$ be the endpoints of $I_n$

For bounded closed $I$:

$J_n:=\left(a_n-\dfrac{\varepsilon}{2^{n+2}},b_n+\dfrac{\varepsilon}{2^{n+1}}\right)$, clearly, $\{J_n\}$ is an open cover of $I$, hence it has a finite subcover $\{J_n\}_{n=1}^m$.

Clearly, the total length of this subcover is greater than the length of $I\implies$\\
$$l(I)\le\displaystyle\sum_{n=1}^m l(J_n)\le\sumninf l(J_n)=\sumninf l(I_n)+\dfrac{\varepsilon}{2}\le m^*(I)+\dfrac{\varepsilon}{2}+\dfrac{\varepsilon}{2}=m^*(I)+\varepsilon$$

For bounded non-closed $I$, suppose $I=(a,b),\forall\varepsilon>0$, then:
$$l((a,b))=l\left(\left[a+\dfrac{\varepsilon}{2},b-\dfrac{\varepsilon}{2}\right]\right)+\varepsilon\le m^*\left(\left[a+\dfrac{\varepsilon}{2},b-\dfrac{\varepsilon}{2}\right]\right)+\varepsilon\le m^*((a,b))+\varepsilon$$
For unbounded $I$, a finite cover does not cover a subinterval of great enough length of $I$, hence does not cover $I$. $\square$\\
\textbf{Theorem:} Sub-additivity
$$\forall \{E_n\}:m^*\left(\infcup E_n\right)\le\sumninf m^*(E_n)$$
\textbf{Proof:}\\
If $\sumninf m^*(E_n)=\infty$ the the proposition is already true. So, suppose $\sumninf m^*(E_n)\in\R^{\geq0}$, then
$$\forall\varepsilon,\exists\{I_k^n\}\st E_n\subset\bigcup_{k=1}^\infty I_k^n\land\sum_{k=1}^\infty l(I_k^n)\le m^*(E_n)+\frac{\varepsilon}{2^n}\implies$$
$$\sumninf m^*(E_n)+\varepsilon\geq\sumninf\sum_{k=1}^\infty l(I_k^n)=\sum_{n,k\in\N^+}l(I_n^k)\geq m^*\left(\infcup E_n\right)$$
\textbf{Note:} The outer measure is translation invariant because set containment is translation invariant
\subsection{Lebesgue-Measurable Sets and Lebesgue Measure}
\textbf{Definition: Lebesgue Measure}\\
A set $E\subset\R$ is Lebesgue measurable if $\forall A\subset\R:m^*(A)=m^*(A\cap E)+m^*(A\cap E^c)$, we write $E\in\mathcal{M}$\\
Clearly, $A=(A\cap E)\cup(A\cap E^c)\implies m^*(A)\le m^*(A\cap E)+m^*(A\cap E^c)$.\\
Hence, $E\in\mathcal{M}\iff m^*(A)\geq m^*(A\cap E)+m^*(A\cap E^c)$\\
\textbf{Corollary:}
\begin{enumerate}
    \item Any null set is measurable
    \item Any interval is measurable
\end{enumerate}
\textbf{Proof:}
\begin{equation}
\begin{split}
    1.\,&\forall\mbox{ Null }N:m^*(N)=0\implies\forall A\subset\R:\\
    &\,A\cap N\subset N\implies m^*(A\cap N)\le m^*(N)=0\\
    &\,A\cap N^c\subset A\implies m^*(A\cap N^c)\le m^*(A)\\
    2.\,&\mbox{Let }I=[a,b]\mbox{ be an interval, }\forall A\subset\R,\forall\varepsilon>0,\exists\{I_n\}\st A\subset\infcup I_n\land m^*(A)\le\sumninf l(I_n)\le m^*(A)+\varepsilon\\
    &\,I_n':=I_n\cap[a,b]\implies A\cap[a,b]\subset\infcup I_n'\implies m^*(A\cap[a,b])\le\sumninf l(I_n')\\
    &\,I_n'':=I_n\cap(-\infty,a),I_n''':=I_n\cap(b,+\infty)\implies A\cap[a,b]^c\subset\infcup(I_n''\cup I_n''')\implies m^*(A\cap[a,b]^c)\le\sumninf l(I_n'')+l(I_n''')\\
    &\implies m^*(A\cap[a,b])+m^*(A\cap[a,b]^c)\le\sumninf l(I_n')+l(I_n'')+l(I_n''')=\sumninf l(I_n)\le m^*(A)+\varepsilon\quad\square
\end{split}
\end{equation}
The second proof holds as $I_n',I_n'',I_n'''$ are all intervals, thus the proof can be generalised slightly.\\
\textbf{Corollary:}
\begin{equation}
\begin{split}
    (1)\,&\R\in\mathcal{M}\\
    (2)\,&E\in\mathcal{M}\implies E^c\in\mathcal{M}\\
    (3)\,&\forall n\in\N^+:E_n\in\mathcal{M}\implies\infcup E_n\in\mathcal{M}\\
    \mbox{(Countable Additivity)}\,&\forall m\neq n:E_m\cap E_n=\emptyset\implies m^*\left(\infcup E_n\right)=\sumninf m^*(E_n)
\end{split}
\end{equation}
Conditions (1)-(3) define a $\sigma$-field, a $\sigma$-field with countable additivity is a \textit{\textbf{measure space}}\\
\textbf{Proof:}
\begin{equation}
\begin{split}
    (1)\,&\forall A\subset\R:m^*(A)=m^*(A)+m^*(\emptyset)=m^*(A\cap\R)+m^*(A\cap\R^c)\\
    (2)\,&\forall A\subset\R:m^*(A)=m^*(A\cap (E^c)^c)+m^*(A\cap E^c)\\
    (\mbox{C.A.})\,&\mbox{Initial Case:}\\
    &\quad\mbox{Suppose }E_1\cap E_2=\emptyset,E_1,E_2\in\mathcal{M}\mbox{, then, }\\
    &\quad\quad(a)\,m^*(A)=m^*(A\cap E_1)+m^*(A\cap E_1^c)\\
    &\quad\quad(b)\,m^*(A\cap E_1^c)=m^*(A\cap E_1^c\cap E_2)+m^*(A\cap E_1^c\cap E_2^c)=m^*(A\cap E_2)+m^*(A\cap(E_1\cup E_2)^c)\\
    &\quad\quad(c)\,m^*(A\cap E_1)+m^*(A\cap E_2)\geq m^*((A\cap E_1)\cup(A\cap E_2))=m^*(A\cap(E_2\cup E_2))\quad(\mbox{sub-additivity})\\
    &\quad\mbox{Substitute }(b)\mbox{ into }(a):\\
    &\quad\quad(d)\,m^*(A)=m^*(A\cap E_1)+m^*(A\cap E_2)+m^*(A\cap(E_1\cup E_2)^c)\\
    &\quad\mbox{Substitute }(c)\mbox{ into }(d):m^*(A)\geq m^*(A\cap(E_1\cup E_2))+m^*(A\cap(E_1\cup E_2)^c)\\
    &\quad\mbox{Sub-additivity}\implies m^*(A)=m^*(A\cap(E_1\cup E_2))+m^*(A\cap(E_1\cup E_2)^c)\\
    &\quad\mbox{Substitute }A=E_1\cup E_2\mbox{ into }(d):m^*(E_1\cup E_2)=m^*(E_1)+m^*(E_2)\\
    &\mbox{General Case:}\\
    &\quad m^*(A)=m^*(A\cap E_1)+m^*(A\cap E_1^c)\\
    &\quad m^*(A)=m^*(A\cap E_1)+m^*(A\cap E_2)+m^*(A\cap(E_1\cup E_2)^c)\\
    &\quad\quad\vdots\\
    &\quad m^*(A)=\sumkn m^*(A\cap E_1)+m^*\left(A\cap\left(\bigcup_{k=1}^nE_k\right)^c\right)\quad\mbox{(Prove this as an exercise)}\\
    &\quad\mbox{We also have:}\left(\bigcup_{k=1}^nE_k\right)^c\supset\left(\bigcup_{k=1}^\infty E_k\right)^c\implies m^*(A)\geq\sum_{k=1}^nm^*(A\cap E_k)+m^*\left(A\cap\left(\bigcup_{k=1}^\infty E_k\right)^c\right)\\
    &\quad\mbox{Take limit on both sides}:\quad(\mbox{the second inequality and second equation hold due to sub-additivity})\\
    &\quad m^*(A)\geq\sum_{k=1}^\infty m^*(A\cap E_k)+m^*\left(A\cap\left(\bigcup_{k=1}^\infty E_k\right)^c\right)\geq m^*\left(A\cap\left(\bigcup_{k=1}^\infty E_k\right)\right)+m^*\left(A\cap\left(\bigcup_{k=1}^\infty E_k\right)^c\right)\\
    &\quad\implies m^*(A)=\sum_{k=1}^\infty m^*(A\cap E_k)+m^*\left(A\cap\left(\bigcup_{k=1}^\infty E_k\right)^c\right)=m^*\left(A\cap\left(\bigcup_{k=1}^\infty E_k\right)\right)+m^*\left(A\cap\left(\bigcup_{k=1}^\infty E_k\right)^c\right)\\
    &\quad\mbox{Substitute }A=\bigcup_{j=1}^\infty E_j\mbox{ into the first equality}\implies m^*\left(\infcup E_n\right)=\sumninf m^*(E_n)\\
    (3)\,&\mbox{Initial Case:}\\
    &\quad m^*(A)=m^*(A\cap E_1)+m^*(A\cap E_1^c),m^*(A\cap E_1^c)=m^*(A\cap E_1^c\cap E_2)+m^*(A\cap E_1^c\cap E_2^c)\implies\\
    &\quad m^*(A)=m^*(A\cap E_1)+m^*(A\cap E_1^c\cap E_2)+m^*(A\cap E_1^c\cap E_2^c)\\
    &\quad\mbox{By sub-additivity}:m^*(A)\geq m^*(A\cap(E_1\cup E_2))+m^*(A\cap(E_1\cup E_2)^c)\\
    &\mbox{General Case:}\\
    &\quad(a)\,E_1,\dots,E_n\in\mathcal{M}\implies E_1\cup\dots\cup E_n\in\mathcal{M}\quad\mbox{(prove by induction with the initial case)}\\
    &\quad(b)\,E_1,E_2\in\mathcal{M}\implies E_1\cap E_2\in\mathcal{M}\quad\mbox{(prove with de Morgan's law)}\\
    &\quad F_1:=E_1\\
    &\quad F_2:=E_2\symbol{92}E_1=E_2\cap E_1^c\\
    &\quad F_3:=E_3\symbol{92}(E_1\cup E_2)=E_2\cap(E_1\cup E_2)^c\\
    &\quad\quad\vdots\\
    &\quad F_k:=E_k\symbol{92}(E_1\cup\dots\cup E_{k-1})=E_k\cap(E_1\cup\dots\cup E_{k-1})^c\\
    &\quad\mbox{Clearly, }\bigcup_{k=1}^\infty F_k\subset\bigcup_{k=1}^\infty E_k,\forall a\in\bigcup_{k=1}^\infty E_k,m_a:=\min\{k\mid a\in E_k\}\implies a\in F_{m_a}\implies \bigcup_{k=1}^\infty E_k\subset\bigcup_{k=1}^\infty F_k\\
    &\quad\implies \bigcup_{k=1}^\infty E_k=\bigcup_{k=1}^\infty F_k\in\mathcal{M}\\
\end{split}
\end{equation}
\textbf{Corollary:}\\
If $\forall n\in\N^+:E_n\in\mathcal{M}$, then $E:=\capninf E_n\in\mathcal{M}$\\
\textbf{Definition:}\\
$\forall E\in\mathcal{M}$, write $m(E)$ instead of $m^*(E)$ can call $m(E)$ the Lebesgue Measure of $E$.\\
\textbf{Example: Non-measurable Set}\\
Define equivalence relation on $[0,1]:x\sim y$ if $x-y\in[-1,1]$ is a rational number. This relation partitions $[0,1]$ into disjoint equivalence classes $(A_\alpha)$ where all elements in the same class differ by a rational number. Each of such classes is countable, hence there are uncountably many classes.\\
Apply the Axiom of Choice to construct set $E\subset[0,1]\st E$ contains exactly one tyelement from each class.\\
Enumerate the rationals in $[-1,1]$ into a sequence $(q_n)$. $E_n:=E+q_n$, clearly, $\infcup E_n\subset[-1,2]$\\
We can see that $(E_n)$ is disjoint, as if $\exists a_\alpha+q_m\in E_m,a_\beta+q_n\in E_n\st a_\alpha+q_m=a_\beta+q_n$, then $a_\alpha-a_\beta=q_n-q_m\in\mathbb{Q}\implies a_\alpha,a_\beta$ are in the same equivalence class$\implies a_\alpha=a_\beta\implies q_n=q_m\implies m=n$\\
$\forall a\in[0,1],\,a$ is in some equivalence class$\implies\exists b\in E\st a\sim b\implies a=b+q_k\in E_k\implies [0,1]\subset\infcup E_n\subset[-1,2]$
$$\implies1=m^*([0,1])\le\sumninf m^*(E_n)=\sumninf m^*(E)\le3$$
However, the sum must be $0$ or $\infty$, hence $E$ must not be measurable.
\subsection{Basic Properties of Lebesgue Measure}
\textbf{Corollary:} $A\in\mathcal{M}\land m(A\Delta B)=0\implies B\in\mathcal{M}\land m(A)=m(B)$\\
\textbf{Proof:}\\
$A\cap B^c,A^c\cap B\subset A\Delta B\implies A\cap B^c,A^c\cap B$ are Null, hence Lebesgue measurable with measure 0.
\begin{equation}
\begin{split}
    0&=m(A\cap B^c)=m(A\cap (A\cap B)^c)=m(A)-m(A\cap B)\implies m(A)=m(A\cap B)\\
    0&=m(B\cap A^c)=m(B\cap (A\cap B)^c)=m(B)-m(A\cap B)\implies m(B)=m(A\cap B)
\end{split}
\end{equation}
$\implies m(A)=m(B)$\\
\textbf{Theorem:}
\begin{equation}
\begin{split}
    (1)\,&\forall\varepsilon>0,\forall A\in\R,\exists\mbox{ open }O\st A\subset O\land0\le m(O)-m^*(A)\le\varepsilon\\
    &\mbox{Consiquently, if }A\in\mathcal{M},\mbox{ then }0\le m(O\symbol{92}A)\le\varepsilon\\
    (2)\,&\forall A\in\R,\exists\mbox{ open sequence }\{O_n\}\st A\subset\infcup O_n\land m\left(\infcap O_n\right)=m^*(A)
\end{split}
\end{equation}
\textbf{Proof:}
\begin{equation}
\begin{split}
    (1)\,&\mbox{By definition of the outer measure, }\exists\mbox{ sequence of intervals }\{I_n\}\st A\subset\infcup I_n\mbox{ and }\sumninf l(I_n)-\frac{\varepsilon}{2}\le m^*(A)\\
    &\mbox{Suppose the endpoints of }I_n\mbox{ are }a_n,b_n\mbox{, then }I_n\subset\left(a_n-\frac{\varepsilon}{2^{n+2}},b_n+\frac{\varepsilon}{2^{n+2}}\right)=:J_n, O:=\infcup J_n\mbox{ is open, then:}\\
    &A\subset O\mbox{ and }m(O)\le\sumninf l(J_n)=\sumninf l(I_n)+\frac{\varepsilon}{2}\le m^*(A)+\varepsilon\\
    &\quad\mbox{If }m(E)<\infty\mbox{, then }0\le m(O\cap A)+m(O\cap A^c)-m(A)\le\varepsilon\implies0\le m(O\cap A^c)\le\varepsilon\\
    &\quad\mbox{If }m(E)=\infty\mbox{, then }E_n:=E\cap(-n,n)\mbox{ and }m(E_n)\mbox{ is finite}\implies\exists\mbox{ open }O_n\supset E_n\st m(O_n\symbol{92}E_n)\le\frac{\varepsilon}{2^n}\\
    &\quad O:=\infcup O_n\mbox{ is open and contains }E\implies O\symbol{92}E=\left(\infcup O_n\right){\symbol{92}}\left(\infcup E_n\right)\subset\infcup(O_n\symbol{92}E_n)\implies\\
    &\quad m(O\symbol{92}E)\le\sumninf m(O_n\symbol{92}E_n)\le\varepsilon\\
    (2)\,&\mbox{Trivially obtainable from (1)}
\end{split}
\end{equation}
\textbf{Theorem:} Suppose $\forall n:A_n\in\mathcal{M}$, then
\begin{equation}
\begin{split}
    (1)\,&\forall n:A_n\subset A_{n+1}\implies m\left(\infcup A_n\right)=\limninf m(A_n)\\
    (2)\,&\forall n:A_n\supset A_{n+1}\mbox{ and }m(A_1)\le\infty\implies m\left(\infcap A_n\right)=\limninf m(A_n)\\
\end{split}
\end{equation}
\textbf{Proof:}
\begin{equation}
\begin{split}
    (1)\,&B_1:=A_1,\forall i>1:B_i:=A_i\symbol{92}A_{i-1}\mbox{, then }\bigcup_{i=1}^\infty B_i=\bigcup_{i=1}^\infty A_i\mbox{, and }\{B_i\}\mbox{ is pairwise disjoint }\implies\\
    &m\left(\cupiinf A_i\right)=m\left(\cupiinf B_i\right)=\sumiinf m(B_i)=\limninf\sumin m(B_i)=\limninf m\left(\cupin B_i\right)=\limninf m(A_n)\\
    (2)\,&A_1\bcsl A_1=\emptyset\subset A_1\bcsl A_2\subset\dots\subset A_1\bcsl A_n\subset\cdots,\mbox{ by (1)}:m\left(\cupninf(A_1\bcsl A_n)\right)=\limninf m(A_1\bcsl A_n)\\
    &\mbox{Since }m(A_1)\mbox{ is finite, }m(A_-\bcsl A_n)=m(A_1)-m(A_n),\mbox{ also, }\cupninf(A_1\bcsl A_n)=A_1\bcsl\cupninf A_n\implies\\
    &m\left(\infcup (A_1\bcsl A_n)\right)=m(A_1)-m\left(\infcup A_n\right)=m(A_1)-\limninf m(A_n)
\end{split}
\end{equation}
\textbf{Theorem:} Any measure $m^*$ on a measure space satisfy:
\begin{equation}
\begin{split}
    (1)\,&\forall n\in\N^+:m^*\left(\cupin A_i\right)=\sumin m^*(A_i)\quad(\mbox{finite additivity})\\
    (2)\,&\forall\{B_n\}\st B_n\supset B_{n+1}\mbox{ and }\capninf B_n=\emptyset:\limninf m(B_n)=0\quad(\mbox{continuity at }\emptyset)
\end{split}
\end{equation}
\textbf{Proof:}
\begin{equation}
\begin{split}
    (1)\,&\mbox{Countable additivity}\implies m^*\left(\cupin A_i\bigcup_{i=n+1}^\infty\emptyset\right)=\sumin m^*(A_i)+\sum_{i=n+1}^\infty m^*(\emptyset)\\
    (2)\,&A_n:=B_n\bcsl B_{n+1}\mbox{ defines a disjoint sequence in }\mathcal{M}\st\cupninf A_n=B_1\\
    &\mbox{Suppose }m^*(B_1)<\infty\implies m^*(A_n)=m^*(B_n)-m(B_{n+1})\geq0\implies\\
    &m^*(B_1)=\sumninf m^*(A_n)=\limkinf\sum_{n=1}^k[m^*(B_n)-m^*(B_{n+1})]=m^*(B_1)-\limninf m^*(B_n)
\end{split}
\end{equation}
\subsection{Borel Sets}
\textbf{Theorem:} The intersection of a family of $\sigma$-fields is a $\sigma$-field\quad(trivially true)\\
\textbf{Definition:} $\mathcal{B}:=\bigcap\{\mathcal{F}\mid \mathcal{F}$ is a $\sigma$-field containing all intervals$\}$\\
In general, we say $\mathcal{G}$ is a $\sigma$-field generated by a family of sets $\mathcal{A}$ if $\mathcal{G}=\bigcap\{\mathcal{F}\mid\mathcal{F}$ is a $\delta$-field such that $\mathcal{A}\subset\mathcal{F}\}$\\
\textbf{Note:} What sets are Borel?\\
\null\hfill{Missing content}\\
\textbf{Theorem:}\\
The Borel set does not change if instead of all intervals, we take all intervals in the form $(a,\infty)$, or $[a,\infty)$, or $(a,b)$, or $[a,b)$ etc.\quad(denote the family of all intervals with $\mathcal{I}$)\\
\textbf{Proof:}\\
Consider the $\sigma$-field generated by the family of all open intervals $\mathcal{J}$, $\mathcal{C}:=\bigcap\{\mathcal{F}\mid\mathcal{F}\supset J,\mathcal{F}$ is a $\sigma$-field$\}$
$$\mathcal{J}\subset\mathcal{I}\implies(\mathcal{F}\subset\mathcal{I}\implies\mathcal{F}\supset\mathcal{J})\implies\mathcal{B}\subset\mathcal{C}$$
Countable intersection of open intervals can generals closed intervals, countable union of closed intervals can generate open intervals. Countably many half close half open intervals can cover infinity.\quad$\square$\\
\textbf{Corollary:} Not all Null sets are Borel\\
\textbf{Proof:}\\
We assume all Null sets to be Borel, then:\\
As $\forall A\subset\R,\exists$ Borel set $O\st A\subset O$ and $O\bcsl A$ is Null$\implies A=O\cap(O\bcsl A)^c$ is Borel.\\
However, as all Borel sets are measurable, and as there exist unmeasurable sets, there must exist non-Borel sets.\\
\textbf{Definition: Completeness}\\
A measure space $({\bf{X}},\mathcal{F},\mu)$ is complete if $\forall F\in\mathcal{F}\st\mu(F)=0:N\subset F\implies N\in\mathcal{F}$\\
A completion of $\mathcal{G}$ is $\{G\cap N\mid G\in\mathcal{F}\land N\subset F\in\mathcal{F}\st \mu(F)=0\}$\\
\textbf{Theorem:} $\mathcal{M}$ is the completion of $\mathcal{B}$\\
\textbf{Proof:}\\
Let the completion of $\mathcal{B}$ be $\mathcal{C}$\\
Suppose $B$ is Null, then $\forall N\subset B\in\mathcal{B}:\forall A\subset\R$ there is $m^*(A\cap N)=0$ and $m^*(A\cap N^c)\le m^*(A)$\\
$\implies m^*(A\cap N)+m^*(A\cap N^c)\le m^*(A)\implies N\in\mathcal{M}\implies\mathcal{C}\subset\mathcal{M}$\\
$\forall E\in\mathcal{M}\st m(E)\le\infty,\exists B\in\mathcal{B}\st E\subset B$ and $m(B)=m(E),m(B\bcsl E)=0\implies\exists L\in\mathcal{B}\st B\bcsl E\subset L\land m(L)=0$\\
$E=B\bcsl(B\bcsl E)=(B^c\cup(B\bcsl E))^c,B\in\mathcal{B}\implies B^c\in\mathcal{B}\implies B^c\cup(B\bcsl E)\in\mathcal{C}\implies (B^c\cup(B\bcsl E))^c\in\mathcal{C}$\\
If $m(E)=\infty$, then $E_n:=E\cap[-n,n]$ is finite$\implies E_n\in\mathcal{C}\implies E=\cupninf E_n\in\mathcal{C}\implies\mathcal{M}\subset\mathcal{C}\implies\mathcal{M}=\mathcal{C}$\\
\textbf{Theorem:} outer approximation
\begin{equation}
\begin{split}
    (1)\,&\forall\varepsilon>0,\forall A\in\R,\exists\mbox{ closed }F\st A\supset F\land0\le m^*(A)-m(F)\le\varepsilon\\
    &\mbox{Consiquently, if }A\in\mathcal{M},\mbox{ then }0\le m(A\bcsl F)\le\varepsilon\\
    (2)\,&\forall A\in\R,\exists\mbox{ closed sequence }\{F_n\}\st A\supset\infcup F_n\land m\left(\infcap F_n\right)=m^*(A)
\end{split}
\end{equation}
\textbf{Proof:} Take the complement of $A$ and use the inner approximation with open sets.\\
\textbf{Definition: Regular Borel Measure}\\
A non-negative countably additive set function $\mu$ on $\mathcal{B}\st\forall B\in\mathcal{B}:$
\begin{equation}
\begin{split}
    &\mu{B}=\inf\{\mu(O)\mid O\mbox{ open, }O\supset B\}\\
    &\mu{B}=\inf\{\mu(F)\mid F\mbox{ closed, }F\subset B\}
\end{split}
\end{equation}
\subsection{Probability}
\subsubsection{Probability Space}
\subsubsection{Events: Conditioning and Independence}
\subsubsection{Applications to Mathematical Finance}
\subsection{Proofs of Propositions}
\clearpage
\section{Measurable Functions}
\subsection{Lebesgue-Measurable Functions}
\textbf{Definition:} $f:E\to\R$ is Lebesgue (Borel) measurable if $\forall$ interval $I\subset\R:f^{-1}(I):=\{x\in\R\mid F(x)\in I\}\in\mathcal{M}\,(\mathcal{B})$
\textbf{Corollary:} $f$ is measurable$\iff\forall a,f^{-1}(\mbox{some type of interval})$ is measurable\\
\textbf{Proof:}\\
$\implies$ is trivial, now we prove the other direction.
\begin{equation}
\begin{split}
    &f^{-1}((-\infty,b))=f^{-1}\left(\cupninf\left(-\infty,b-\frac{1}{n}\right]\right)=\cupninf f^{-1}\left(\left(-\infty,b-\frac{1}{n}\right]\right)\quad(\mbox{closed}\iff\mbox{open})\\
    &f^{-1}((-\infty,a])=f^{-1}(\R\bcsl(a,\infty))=E\bcsl f^{-1}(a,\infty)\quad(\mbox{changing side of infinity})\\
    &f^{-1}(a,b)=f^{-1}((-\infty,b)\cap(a,\infty))=f^{-1}((-\infty,b))\cap f^{-1}((a,\infty))\quad(\mbox{infinite to finite})\\
    &\mbox{Finite to infinite is trivial}\\
\end{split}
\end{equation}
Hence, from any type of interval, we can generate all other intervals.\quad$\square$\\
\textbf{Examples:}\\
It can be easiy proven that all: continuous funcitons, indicator functions, and monotone functions are measurable.
\subsection{Properties}
\textbf{Theorem:} $f,g$ are measurable (Borel)$\implies f+g,fg$ are measurable (Borel)\\
\textbf{Proof:}\\
Take some sequance of all rationals $\{q_n\}$, then
$$\forall a\in\R,B_a:=\cupninf\{t\mid f(t)<q_n\land g(t)<a-q_n\}=\cupninf(\{t\mid f(t)<q_n\}\cap\{t\mid g(t)<a-q_n\})\in\mathcal{M}$$
$$t\in (f+g)^{-1}((-\infty,a))\implies f(t)+g(t)<a\implies f(t)<a-g(t)\implies\exists q_m\st f(t)<q_m<a-g(t)\implies$$
$$f(t)<q_m\land g(t)<a-q_m\implies t\in B_a\implies B_a\supset(f+g)^{-1}((-\infty,a)),\mbox{ inclusion in the other direction is trivial}$$
$$\implies (f+g)^{-1}((-\infty,a))=B_a\in\mathcal{M}$$
Suppose $h:E\to\R$ is measurable, then $\left(h^2\right)^{-1}((a,\infty))=\{x\mid h^2(x)>a\}=\{x\mid h(x)>\sqrt{a}\}\cup\{x\mid h(x)<-\sqrt{a}\}\in\mathcal{M}$
$$\implies fg=\frac{1}{4}\left((f+g)^2-(f-g)^2\right)\in\mathcal{M}$$
\textbf{Examples:}\\
Let $f$ be a measurable function and $A\in\mathcal{M}$, then $f\cdot\1_A$ is measurable. $A:=\{x\in E\mid f(x)>0\}$, then\\
$f^+(x)=f\cdot\1_A=\begin{cases}
    f(x)&f(x)>0\\
    0&f(x)\le0
\end{cases}$ is measurable, similarly is $f^-(x)=\begin{cases}
    0&f(x)>0\\
    -f(x)&f(x)\le0
\end{cases}$\\
It can also be shown that $f^+(x),f^-(x)$ is measurable$\iff f(x)$ is measurable.\\
\textbf{Lemma:} A stronger and more general result\\
Suppose $F:\R^2\to\R$ is continuous, if $f,g$ are measurable, then $h(x)=F(f(x),g(x))$ is measurable\\
\textbf{Proof:}\\
$\forall a\in\R,G_a:=\{(u,v)\mid F(u,v)>a\}=F^{-1}(a,+\infty)$ is open$\implies G_a=\cupninf((p_n,q_n)\times(r_n,s_n))$
$$\forall (p,q)\times(r,s):\{x\mid (f(x),g(x))\in(p,q)\times(r,s)\}=\{x\mid f(x)\in(p,q)\}\cap\{x\mid g(x)\in(r,s)\}\in\mathcal{M}$$
$$\forall a\in\R,\{x\mid h(x)>a\}=\{x\mid (f(x),g(x))\in G_a\}=\cupninf(\{x\mid f(x)\in(p,q)\}\cap\{x\mid g(x)\in(r,s)\})\in\mathcal{M}$$
\textbf{Theorem:}\\
$\{f_n\}$ is a measurable sequence$\implies\max\limits_{n\le k}f_n,\sup\limits_{n\in\N}f_n,\limninf\sup f_n$ are measurable (as well as their counterparts)\\
\textbf{Proof:} $\forall k\in\N:$
\begin{equation}
\begin{split}
    &\{x\mid(\max_{n\le k}f_n)(x)>a\}=\cupnk\{x\mid f_n(x)>a\}\in\mathcal{M}\\
    &\{x\mid(\sup_{n\geq k}f_n)(x)>a\}=\bigcup_{n=k}^\infty\{x\mid f_n(x)>a\}\in\mathcal{M}\\
    &\limninf\sup f_n=\inf_{n\geq1}\{\sup_{m\geq n}f_m\}\mbox{ is measurable}
\end{split}
\end{equation}
\textbf{Corollary:} $\{f_n\}$ is a measurable sequence and $f_n\to f\implies f$ is measurable\\
\textbf{Proof:} $\limninf f_n=\limninf\sup f_n$, thus clearly.\\
\textbf{Theorem:} If $f:E\to\R$ is measurable, $g:E\to\R\st\{x\mid f(x)\neq g(x)\}$ is null, then $g$ is measurable\\
\textbf{Proof:}\\
$d(x):=g(x)-f(x)$ is equal to $0$ except on a null set$\implies\{x\mid d(x)>a\}=\begin{cases}
    \mbox{a null set}&a\geq0\\
    \mbox{a full set (complement of null)}&a<0
\end{cases}$\\
$\implies d(x)$ is measurable$\implies f=f+d$ is measurable\quad$\square$\\
\textbf{Corollary:} Suppose $\{f_n\}$ is a measurable sequence and $A$ is null s.t. $\forall x\in E\bcsl A:f_n(x)\to f$, then $f$ is measurable\\
\textbf{Proof:} $\forall x\in\R:\1_{A^c}\cdot f_n(x)\to\1_{A^c}\cdot f(x)$ hence is measurable$\implies f$ is measurable\\
\textbf{Corollary:} $\{f_n(x)\}$ is measurable sequence$\implies E=\{x\mid f_n(x)\mbox{ converges}\}\in\mathcal{M}$\\
\textbf{Proof:} $g(x):=\lim\sup f_n(x),h(x):=\lim\inf f_n(x)\implies g(x)-h(x)\mbox{ is measurable}\implies(g(x)-h(x))^{-1}([0,0])\in\mathcal{M}$\\
\textbf{Definition:} Suppose $f:E\to\overline\R$ is measurable, the \textit{\textbf{essential supremum}} and \textit{\textbf{essential infremum}}
\begin{equation}
\begin{split}
    \esssup:=\inf\{z\mid f\le z\alev\}\\ \essinf:=\sup\{z\mid f\geq z\alev\}
\end{split}
\end{equation}
$$A:=\{x\mid f(x)>\esssup f\},A_n:=\{x\mid f(x)>\esssup f+\frac{1}{n}\},\,\clear A=\cupninf A_n$$
\textbf{Proposition:} $\esssup f+\esssup g\geq\esssup(f+g)$\\
\textbf{Proof:}\\
$f\le\esssup f,g\le\esssup g\alev\implies f+g\le\esssup f+\esssup g\alev\implies\esssup f+\esssup g\in\{z\mid f+g\le z\alev\}$\\
$\implies\esssup(f+g)=\inf\{z\mid f+g\le z\alev\}\le\esssup f+\esssup g$\\
\textbf{Corollary:} $\esssup f\le\sup f$ clearly\\
\textbf{Corollary:} $f$ is continuous$\implies\esssup f=\sup f$\\
\textbf{Proof:}\\
Suppose $f$ is continuous and $\esssup f\neq\sup f$, then $\exists N$ null$\st\forall x\in N: f(x)>\esssup f$, choose an arbitrary $x_0\in N$, then $\exists\delta\st\forall x\in(x_0-\delta,x_0+\delta):f(x)\geq\esssup f\implies (x_0-\delta,x_o+\delta)\in N\mbox{ is not null (contradiction)}$
\subsection{Probabilities}
\clearpage
\section{Integral}
\subsection{Definition of the Integral}
\textbf{Definition: Simple Function}\\
$\varphi:R\to\{a_1,\dots,a_n\}\st\forall i:A_i:=\varphi^{-1}(\{a_i\})\in\mathcal{M},\varphi(x)=\sumin a_i\1_A(x)$, thus simple functions are measurable\\
\textbf{Definition:} The \textit{\textbf{Lebesgue integral}} over $E\in\mathcal{M}$ of the simple function $\varphi$ is $dint_E\varphi \dif m=\sumin a_im(A_i\cap E)$\\
\textbf{Example:} $\dint_\R\1_\Q \dif m=1\times m(\Q)+0\times m(\R\bcsl\Q)=0$\\
\textbf{Definition:} $\forall$ non-negative measurable unctions $f$ and $E\in\mathcal{M}$
$$\dint_Ef\dif m:=\sup Y(E,f)\mbox{, where }Y(E,f):=\left\{\dint_E\varphi \dif m\,\middle\vert\,0\le\varphi\le f\land\varphi\mbox{ is simple}\right\}$$
If $E=[a,b]$, we write the integral as: $\dint_a^b f\dif m$ or $\dint_a^b f\dif m(x),\dint f\dif m:=\dint_\R f\dif m,\dint_A f\dif m:=\dint\1_A\dif m$\\
\textbf{Theorem:} Let $\varphi,\psi$ be simple functions, then
\begin{equation}
\begin{split}
    (1)\,&\varphi\le\psi\implies\int_E\varphi \dif m\le\int_E\psi \dif m\\
    (2)\,&A\cap B=\emptyset\land A,B\in\mathcal{M}\implies\int_{A\cup B}\varphi \dif m=\int_A\varphi \dif m+\int_B\varphi \dif m\\
    (3)\,&\forall a>0:\int_Ea\cdot\varphi \dif m=a\cdot\int_E\varphi \dif m
\end{split}
\end{equation}
\textbf{Proof:}
\begin{equation}
\begin{split}
    (1)\,&\mbox{clearly}\\
    (2)\,&\int_{A\cup B}\varphi \dif m=\sum c_im(D_i\cap(A\cup B))=\sum c_1(m(D_i\cap A)+m(D_i\cap B))\\
    &=\sum c_im(D_i\cap A)+\sum c_im(D_i\cap B)=\int_A\varphi \dif m+\int_B\varphi \dif m\\
    (3)\,&\int_Ea\cdot\varphi \dif m=\sum a\cdot c_im(E\cap A_i)=a\sum c_im(E\cap A_i)=a\cdot\int_E\varphi \dif m
\end{split}
\end{equation}
\textbf{Theorem:} Suppose $f,g$ are non-negative measurable functions, then
\begin{equation}
\begin{split}
    (1)\,&A\in\mathcal{M}\land f\le g\implies\int_Af\dif m\le\int_Ag\dif m\\
    (2)\,&A,B\in\mathcal{M}\land B\subset A\implies\int_Bf\dif m\le\int_Af\dif m\\
    (3)\,&a\geq0\implies\int_Aa\cdot f\dif m=a\int_Af\dif m\\
    (4)\,&A\mbox{ is null}\implies\int_Af\dif m=0\\
    (5)\,&A,B\in\mathcal{M}\land A\cap B=\emset\implies\int_{A\cup B}f\dif m=\int_Af\dif m+\int_Bf\dif m
\end{split}
\end{equation}
\textbf{Proof:}
\begin{equation}
\begin{split}
    (1)\,&Y(A,f)\subset Y(A,g)\implies\sup Y(A,f)\le\sup Y(A,g)\\
    (2)\,&\forall\mbox{ simple function }\varphi\mbox{ on }B\mbox{ below }f,\varphi^*:=\begin{cases}
        \varphi&x\in B\\
        0&x\in A\bcsl B
    \end{cases}\implies\int_B\varphi \dif m=\int_A\phi^*\dif m\implies Y(B,f)\subset Y(A,f)\\
    (3)\,&\sup Y(A,af)=a\cdot\sup Y(A,f)\quad(\mbox{clearly})\\
    (4)\,&\mbox{clearly}\\
    (5)\,&c\in Y(A\cup B,f)\implies c=\int_{A\cup B}\varphi \dif m=\int_A\varphi \dif m+\int_B\varphi \dif m\implies Y(A\cup B,f)\subset Y(A,f)+Y(B,f)\\
    &\forall\varphi,\psi\le f,\varphi^*:=\begin{cases}
        \varphi&x\in A\\
        0&x\notin A
    \end{cases},\psi^*:=\begin{cases}
        \psi&x\in B\\
        0&x\notin B
    \end{cases}\implies\int_A\varphi \dif m=\int_A\varphi^*\dif m,\int_B\psi \dif m=\int_B\psi^*\dif m\\
    &\gamma:=\varphi^*+\psi^*\implies\int_A\varphi \dif m+\int_B\psi \dif m=\int_A\varphi^*\dif m+\int_B\psi^*\dif m=\int_A\gamma \dif m+\int_B\gamma \dif m=\int_{A\cup B}\gamma \dif m\\
    &\implies Y(A,f)+Y(B,f)\subset Y(A\cup B,f)\implies Y(A\cup B,f)=Y(A,f)+Y(B,f)\\
    &\implies\sup Y(A\cup B,f)=\sup Y(A,f)+\sup Y(B,f)
\end{split}
\end{equation}
\textbf{Proposition:} if $f,g$ are measurable and $f\le g\alev\implies\dint f\dif m\le\int g\dif m$\\
\textbf{Proof:}\\
$A:=\{x\mid f(x)>g(x)\}$, by definition, $A$ is null$\implies f^*:=\begin{cases}
    g&x\in A\\
    f&x\notin A
\end{cases},f_*:=\begin{cases}
    f(x)-g(x)&x\in A\\
    0&x\notin A
\end{cases}\implies$
$$f^*\le g,f=f^*+f_*\implies\int f\dif m=\int f^*+f_*\dif m=\int f^*\dif m+\int f_*\dif m\le\int g\dif m$$
\subsection{Monotone Convergence Theorem}
\textbf{Fatou's Lemma:} $\{f_n\}$ is a non-negative measurable sequence$\implies\limninf\displaystyle\inf\int_Ef_n\dif m\geq\dint_E\limninf\inf f_n\dif m$\\
\textbf{Proof:}\\
$f=\limninf\inf\limits_{k\geq n}f_k$, let $\varphi$ be a simple function$\st\varphi\le f$, choose small $\varepsilon\st\overline{\varphi}(x):=\begin{cases}
    \varphi(x)-\varepsilon>0&\varphi(x)\geq0\land x\in E\\
    0&\varphi(x)=0\lor x\notin E
\end{cases}$
$$A_n:=\{x\mid\inf_{k\geq n}f_k(x)\geq\overline{\varphi}(x)\}\implies A_n\subset A_{n+1},\cupninf A_n=\R\implies$$
$$\int_{A_n\cap E}\overline\varphi \dif m\le\int_{A_n\cap E}\inf_{k\geq n}f_k\dif m\le\int_{A_n\cap E}f_k\dif m\,(k\geq n)\le\int_Ef_k\dif m\,(k\geq n)\implies\int_{A_n\cap E}\overline\varphi \dif m\le\limninf\inf\int_E f_n\dif m$$
$$\mbox{Let }n\to\infty:\overline\varphi=\sum_{i=1}^lc_i\1_{B_i}\implies\int_{A_n\cap E}\overline\varphi \dif m=\sum_{i=1}^l c_im(A_n\cap E\cap B_i)\to\sum_{i=1}^lc_im(E\cap B_i)=\int_E\overline\varphi \dif m$$
$$\implies\int_E\varphi \dif m-\varepsilon\cdot m(\{x\in E\mid\varphi(x)>0\})\le\limkinf\inf\int_Ef_k\dif m$$
Suppose $m(\{x\in E\mid\varphi(x)>0\})\le\infty$, then $\dint_E\varphi \dif m\le\displaystyle\limkinf\inf\dint_Ef_k\dif m$\\
If $m(\{x\in E\mid\varphi(x)\geq0\})=\infty$, then $\dint_E \displaystyle\limkinf\inf f_k\dif m\geq\dint_E\varphi \dif m=\infty$
$$a:=\frac{1}{2}\min\{c_i\},D_n:=\{x\mid\inf_{k\geq n}f_k(x)>a\}\implies\cupninf D_n\supset\{x\in E\mid\varphi(x)>0\}\implies\int_{D_n\cap E}a\dif m\to\infty\implies$$
$$\int_{D_n\cap E}a\dif m\le\int_{D_n\cap E}\inf_{k\geq n}f_k\dif m\le\int_{D_n\cap E}f_k\dif m\le\int_Ef_k\dif m\,(k\geq n)\implies\limninf\inf\int_Ef\dif m=\infty\geq\int_E\limninf\inf f_n\dif m$$
\textbf{Monotone Convergence Theorem (MCT):}\\
If $\{f_n\}$ is a non-negative measurable sequence and $\forall x\in E:f_n(x)\nearrow f(x)$, then $\limninf\dint_Ef_n(x)\dif m=\dint_Ef\dif m$\\
\textbf{Proof:}
$$f_n\le f\implies\int_Ef_n\dif m\le\int_ef\dif m\implies\limninf\sup\int_Ef_n\dif m\le\int_ef\dif m\le\limninf\inf\int_Ef_n\dif m$$
$$\implies\int_Ef\dif m=\limninf\inf\int_Ef_n\dif m=\limninf\sup\int_Ef_n\dif m\quad\square$$
\textbf{Corollary:} Suppose $\{f_n\}$ is non-negative measurable and $f_n\nearrow f\alev$, then $\forall E$ measurable: $\dint_Ef_n\dif m\nearrow\dint_Ef\dif m$\\
\textbf{Proof:}
$$A:=\{x\mid f_n(x)\nearrow f(x)\land x\in E\}\implies g_n:=\begin{cases}
    f_n&x\in A\\
    0&x\in A^c
\end{cases}\nearrow\begin{cases}
    f&x\in A\\
    0&x\in A^c
\end{cases}=:g,E=(E\cap A)\cup(E\cap A^c)\implies$$
$$\int_Eg_n\dif m=\int_{E\cap A}f_n\dif m+\int_{E\cap A^c}=\int_{E\cap A}f_n\dif m+\int_{E\cap A^c}f_n\dif m=\int_Ef_n\dif m,\int_Ef_n\dif m=\int_Ef_n\dif m\nearrow\int_Eg\dif m=\int_Ef\dif m$$
\textbf{Proposition:} $\forall$ non-negative measurable $f$, $\exists$ non-negative measurable sequence $\{s_n\}$ of simple functinos$\st s_n\nearrow$\\
\textbf{Proof:} Consider $s_n:=\displaystyle\sum_{k=0}^{2^{2n}}\dfrac{k}{2^n}\cdot\1_{f^{-1}([\frac{k}{2^n},\frac{k+1}{2^n}])}$
\subsection{Integrable Function}
\textbf{Definition:} If $E\in\mathcal{M}$ and for measurable function $f$, $\dint_Ef^+\dif m$ and $\dint_Ef^-\dif m$ are finite, then $f$ is \textit{\textbf{integrable}}
$$\int_Ef\dif m:=\int_Ef^+\dif m-\int_Ef^-\dif m,\,\mathcal{L}^1(E):=\{f\mid f\mbox{ is integrable on }E\}$$
\textbf{Corollary:} quite trivially, $f\in\mathcal{L}^1\iff|f|\in\mathcal{L}^1$\\
\textbf{Proposition:} If $f,g$ are integrable and $f\le g$, then $\dint f\dif m\le\int g\dif m$\quad(Prove as an exercise)\\
\textbf{Standard procedure for proofs for integral properties:}
\begin{enumerate}
    \item Verifying the property for indicator functions
    \item Use Linearity to extend the result to non-negative simple functions
    \item Use the MCT to show property for non-negative measurable functions
    \item Extend to the class of functions by writing $f=f^+-f^-$ an linearity
\end{enumerate}
\textbf{Theorem:} $\forall$ integrable function $f,g$, $f+g$ is integrable and $\dint_Ef+g\dif m=\dint_Ef\dif m+\dint_Eg\dif m$\\
\textbf{Proof:}
\begin{equation}
\begin{split}
    1.\,&\mbox{Trivial}\\
    2.\,&\varphi=\sum a_i\1_{A_i},\psi=\sum b_j\1_{B_j}\implies\varphi+\psi=\sum_{i,j}(a_i+b_j)\cdot\1_{A_i\cap B_j}\implies\int_E\varphi+\psi\dif m\sum_{i,j}(a_i+b_j)m(A_i\cap B_j\cap E)\\
    &=\sum_ia_i\sum_jm(A_i\cap B_j\cap E)+\sum_jb_h\sum_im(A_i\cap B_j\cap E)=\sum_ia_im(A_i\cap\bigcup_jB_j\cap E)+\sum_jb_hm(\bigcup_iA_i\cap B_j\cap E)\\
    &=\sum_ia_im(A_i\cap E)+\sum_jb_jm(B_j\cap E)=\int_E\varphi\dif m+\int_E\psi\dif m\\
    3.\,&\forall f^*,g^*\mbox{ non-negative measurable, }\exists\{s_n\},\{t_n\}\mbox{ simple }\st s_n\incto f^*,t_n\incto g^*\implies s_n+t_n\incto f^*+g^*\\
    &\mbox{By the MCT}:\int_E f^*+g^*\dif m=\limninf\int_Es_n+t_n\dif m=\limninf\int_Es_n\dif m+\limninf\int_Et_n\dif m=\int_Ef^*\dif m+\int_Eg^*\dif m\\
    4.\,&\forall f,g\mbox{ non-negative integrable}:\int_E|f+g|\dif m\le\int_E|f|+|g|\dif m\\
    &f+g=\begin{cases}
        (f+g)^+-(f+g)^-\\
        (f^+-f^-)+(g^+-g^-)
    \end{cases}\implies(f+g)^++f^-+g^-=f^++g^++(f+g)^-\implies\\
    &\int_E(f+g)^+\dif m+\int_E f^-\dif m+\int_Eg^-\dif m=\int_Ef^+\dif m+\int_Eg^+\dif m+\int_E(f+g)^-\dif m\implies\\
    &\int_E(f+g)^+\dif m-\int_E(f+g)^-\dif m=\int_Ef^+\dif m-\int_Ef^-\dif m+\int_Eg^+\dif m-\int_Eg^-\dif m\\
    &\implies\int_Ef+g\dif m=\int_Ef\dif m+\int_Eg\dif m\\
\end{split}
\end{equation}
\textbf{Proposition:} $f$ is integrable and $c\in\R\implies\dint_Ec\cdot f\dif m=c\cdot\dint_Ef\dif m$
\begin{equation}
\begin{split}
    1.-3.\,&\mbox{Has been done}\\
    4.\,&c\geq0:\begin{cases}
        (c\cdot f)^+=c\cdot(f^+)\\
        (c\cdot f)^-=c\cdot(f^-)
    \end{cases}\mbox{then by definition}\\
    &c<0:\begin{cases}
        (c\cdot f)^+=-c\cdot(f^+)\\
        (c\cdot f)^-=-c\cdot(f^-)
    \end{cases}\mbox{then by definition}
\end{split}
\end{equation}
\textbf{Inference:} $\forall E$ measurable, $\mathcal{L}^1(E)$ is a vector space\\
\textbf{Theorem:}
\begin{equation}
\begin{split}
    (1)\,&\forall A\in\mathcal{M}:\int_Af\dif m\le\int_Ag\dif m\implies f\le g\alev\\
    (2)\,&\forall A\in\mathcal{M}:\int_Af\dif m=\int_Ag\dif m\implies f=g\alev
\end{split}
\end{equation}
\textbf{Proof:}
\begin{equation}
\begin{split}
    (1)\iff(1')\,&\forall A\in\mathcal{M}:\int_Ah\dif m\geq0\implies h\geq0\alev\\
    (1')\,&A_n:=\{x\mid h(x)\le-\frac{1}{n}\},A:=\{x\mid h(x)\le 0\}\implies A=\cupninf A_n\\
    &\implies\forall n:0=\int_{A_n}h\dif m\le\int_{A_n}-\frac{1}{n}\dif m=-\frac{1}{n}m(A_n)\implies m(A_n)=0\\
    &\implies m(A)=m(\cupninf A_n)\le\sumninf m(A_n)=0\\
    (2')\,&\mbox{is similar to }(1')
\end{split}
\end{equation}
\textbf{Proposition:}
\begin{equation}
\begin{split}
    (1)\,&\mbox{integrable}\implies\mbox{finite}\alev\\
    (2)\,&f\mbox{ measurable}\land A\in\mathcal{M}\implies m(A)\cdot\inf_Af\le\int_Af\dif m\le m(A)\cdot\sup_Af\\
    (3)\,&\left|\int f\dif m\right|\le\int|f|\dif m\\
    (4)\,&f\geq0\land\int f\dif m=0\implies f=0\alev
\end{split}
\end{equation}
\textbf{Proof:}
\begin{equation}
\begin{split}
    (1)\,&\mbox{trivial}\\
    (2)\,&m(A)\cdot\inf_Af=\int_A\inf_Af\dif m,\inf_Af\le f\implies m(A)\cdot\inf_Af\le\int_Af\dif m\\
    (3)\,&\begin{cases}
        \dint f\dif m\geq0\implies|\dint f\dif m|=\dint f\dif m,|f|\geq f\implies\dint|f|\dif m\geq\dint f\dif m\\
        \dint f\dif m<0\implies|\dint f\dif m|=-\dint f\dif m,|f|\geq -f\implies\dint|f|\dif m\geq\dint-f\dif m=-\dint f\dif m
    \end{cases}\\
    (4)\,&\mbox{Similar to the last theorem}
\end{split}
\end{equation}
\textbf{Theorem:} $f\geq0\implies A\mapsto\dint_Af\dif m$ is a measure\\
\textbf{Proof:}\\
$\mu(A):=\dint_Af\dif m$, now we only have to prove countable additivity for $mu$. For pair-wise disjoint $E_i:$
$$f\cdot\1_{\cupin E_i}\incto f\cdot \1_{\cupiinf E_i}\implies\int f\cdot\1_{\cupin E_i}\dif m\incto\int f\cdot\1_{\cupiinf E_i}\dif m=\dint_{\cupiinf E_i}f\dif m=\mu(\cupiinf E_i)$$
$$\int f\cdot\1_{\cupin E_i}\dif m=\int_{\cupin E_i}f\dif m=\sumin\int_{E_i}f\dif m\incto\sumiinf\mu(E_i)\implies\mu(\cupiinf E_i)=\sumiinf\mu(E_i)\quad\square$$
\subsection{Dominated Convergence Theorem}
\textbf{Dominated Convergence Theorem (DCT):}\\
Suppose measurable sequence $\{f_n\}$ is bounded by integrable function $g\alev$ on $E$ ($\forall n:|f_n|<g$), then
$$\limninf\dint_Ef_n(x)\dif m=\int_E\limninf f_n\dif m$$
\textbf{Proof:}\\
Suppose $f_n\geq0$, then by Fatou's Lemma:
\begin{equation}
\begin{split}
    (1)\,&\int_E\limninf f_n\dif m=\int_E\limninf\inf f_n\dif m\le\limninf\inf\int_Ef_n\dif m\\
    (2)\,&\int_Eg\dif m-\int_E\limninf f_n\dif m=\int_E\limninf(g-f_n)\dif m=\int_E\limninf\inf(g-f_n)\dif m\\
    &\le\limninf\inf\int_E(g-f_n)\dif m=\int_Eg\dif m-\limninf\sup\int_Ef_n\dif m\implies\limninf\sup\int_Ef_n\dif m\le\int_E\limninf f_n\dif m\\
\end{split}
\end{equation}
Combine (1), (2) and we have the desired result.\\
Now for the general case, we have $0\le f_n+g\le 2g$, applying the previous result to $f_n+g$ and trivially\quad$\square$\\
\textbf{Example:} $f_n(x)=\dfrac{n\sin x}{1+n^2x^{\frac{1}{2}}},E=[0,1]$
$$\forall x\in[0,1]:\limninf f_n(x)=0,\left|\frac{n\sin x}{1+n^2x^{\frac{1}{2}}}\right|\le\frac{n}{n^2x^\frac{1}{2}}\le\frac{1}{x^\frac{1}{2}}\mbox{ is integrable over }E\implies\int_{[0,1]}\limninf f_n(x)\dif m=0$$
\textbf{Proposition:}\\
Suppose $f$ is integrable, $g_n:=f\cdot\1_{[-n,n]},h_n:=\min\{f,n\}$, then $\dint|f-g_n|\dif m\to0,\dint|f-h_n|\dif m\to0$\\
\textbf{Proof:} Trivial application of the DCT\\
\textbf{Example:} $f_n=\limninf\dint_a^\infty\dfrac{n^2xe^{-n^2x^2}}{1+x^2}\dif x\,(a\geq0)$\\
Substitute $u=nx\implies\dint_a^\infty\dfrac{n^2xe^{-n^2x^2}}{1+x^2}\dif x=\dint_{an}^\infty\dfrac{ue^{-u^2}}{1+(\frac{u}{n})^2}\dif u$
\begin{equation}
\begin{split}
    (a=0)\,&\forall u\geq0:\frac{ue^{-u^2}}{1+(\frac{u}{n})^2}\incto ue^{-u^2}\implies\limninf\int_0^\infty\frac{ue^{-u^2}}{1+(\frac{u}{n})^2}\dif u=\dint_0^\infty\limninf\frac{ue^{-u^2}}{1+(\frac{u}{n})^2}\dif u=\int_0^\infty ue^{-u^2}\dif u=\frac{1}{2}\\
    (a>0)\,&0\le\limninf\int_{an}^\infty\frac{ue^{-u^2}}{1+(\frac{u}{n})^2}\dif u\le\limninf\int_{an}^\infty ue^{-u^2}\dif u=0
\end{split}
\end{equation}
\textbf{Example:} $f_n=\limninf\dint_0^\infty\dfrac{1}{(1+\frac{x}{n})^n\cdot x^{\frac{1}{n}}}\dif x$
$$\left(1+\frac{x}{n}\right)^n=\left(\left(1+\frac{x}{n}\right)^{\frac{n}{x}}\right)^x\incto e^x,\begin{cases}
    x^{\frac{1}{n}}\decto1&x>1\\
    x^{\frac{1}{n}}\incto1&x<1
\end{cases}\implies\dfrac{1}{(1+\frac{x}{n})^n\cdot x^{\frac{1}{n}}}\dif x\to\dfrac{1}{e^x}$$
\begin{equation}
\begin{split}
    0<x\le1:&\frac{1}{(1+\frac{x}{n})^n\cdot x^{\frac{1}{n}}}\le\frac{1}{(1+x)\cdot x^{\frac{1}{2}}}<\frac{1}{\sqrt{x}}\\
    x>1\land n\geq2:&\frac{1}{(1+\frac{x}{n})^n\cdot x^{\frac{1}{n}}}<\frac{1}{(1+\frac{x}{n})^n}\le\frac{1}{(1+\frac{x}{2})^2}\\
\end{split}
\end{equation}
$$\implies|f_n|_{(n\geq2)}\le g:=\begin{cases}
    \dfrac{1}{\sqrt{x}}&0<x\le1\\
    \dfrac{1}{(1+\frac{x}{2})^2}&x>1
\end{cases}\implies\limninf\int_0^\infty\dfrac{1}{(1+\frac{x}{n})^n\cdot x^{\frac{1}{n}}}\dif x=\int_0^\infty\frac{1}{e^x}\dif x=1$$
\textbf{Proposition:} for non-negative measurable sequence $\{f_n\}:\dint\sumninf f_n\dif m=\sumninf\dint f_n\dif m$\null\hfill{(Prove as an exercise)}
\textbf{Beppo-Levi Theorem (BLT):}\\
$\sumkinf\dint|f_k|\dif m$ is finite$\implies\sumkinf f_k$ converges a.e and $\dint\sumkinf f_k\dif m=\sumkinf\dint f_k\dif m$\\
\textbf{Proof:}\\
$\dint\sumkinf|f_k|\dif m=\sumkinf\dint|f_k|\dif m$ is finite$\implies\sumkinf|f_k|$ is finite a.e. (hence integrable a.e.)$\implies\sumkinf f_k$ converges a.e.
$$\left|\sumkinf f_k\right|\le\sumkinf|f_k|\implies\sumkinf\dint f_k\dif m=\limninf\dint\sumkn f_k\dif m\xlongequal{\mbox{DCT}}\dint\limninf\sumkn f_k\dif m=\dint\sumkinf f_k\dif m$$
\textbf{Example:}
\begin{equation}
\begin{split}
    (1)\,&\int_0^1\left(\frac{\ln x}{1-x}\right)^2\dif x\xlongequal{\sumkinf kx^{k-1}=\frac{1}{(1-x)^2}}\dint_0^1\sumninf nx^{n-1}(\ln x)^2\dif x\xlongequal{\mbox{BLT}}\sumninf n\dint x^{n-1}(\ln x)^2\dif x=2\sumninf\frac{1}{n^2}=\frac{\pi^2}{3}\\
    (2)\,&\int_0^\infty\frac{x}{e^x-1}\dif x=\int_0^\infty x\cdot\frac{e^{-x}}{1-e^{-x}}\dif x=\int_0^\infty\sumninf xe^{-nx}\dif x=\sumninf\int_0^\infty xe^{-nx}\dif x=\sumninf\frac{1}{n^2}=\frac{\pi^2}{6}
\end{split}
\end{equation}
\subsection{Relation to the Riemann Integral}
\textbf{Theorem:} $f:[a,b]\to\R$, then\\
Riemann Integrable$\iff f$ bounded, and continuous a.e. (Lebesgue)$\implies$ Riemann integral = Lebesgue integral\\
\textbf{Proof:}
$$\forall P:a=x_0<\cdots<x_n=b\mbox{, there is}$$
\begin{equation}
\begin{split}
    \overline{S}(P)&=\sumin M_i\Delta x_i=\int_{[a,b]}\sumin M_i\cdot\1_{[x_{i-1},x_i]}\dif m=:\int_{[a,b]}\overline{s}(P)\dif m\\
    \underline{S}(P)&=\sumin m_i\Delta x_i=\int_{[a,b]}\sumin m_i\cdot\1_{[x_{i-1},x_i]}\dif m=:\int_{[a,b]}\underline{s}(P)\dif m
\end{split}
\end{equation}
$\forall\{P_n\}\st P_{n+1}$ adds new points to $P_n$ and $\max\{\Delta x_i\}\to0$, there is 
$$\underline{s}(P_1)\le\underline{s}(P_2)\le\cdots\le f\le\cdots\le\overline{s}(P_2)\le\overline{s}(P_1)\implies\limninf\underline{s}(p_n)\le f\le\limninf\overline{s}(P_n)$$
$$\implies\begin{cases}
    \limninf\overline{S}(P_n)=\limninf\dint_{[a,b]}\overline{s}(P)\dif m\xlongequal{\mbox{MCT}}\dint_{[a,b]}\limninf\overline{s}(P_n)\dif m\\[8pt]
    \limninf\underline{S}(P_n)=\limninf\dint_{[a,b]}\underline{s}(P)\dif m\xlongequal{\mbox{MCT}}\dint_{[a,b]}\limninf\underline{s}(P_n)\dif m
\end{cases}$$
\null\hfill{(0):\,$f$ is continuous at $x\iff\limninf\underline{s}(P_n)=f(x)=\limninf\overline{s}(P_n)$}\\
$f$ is Riemann integrable$\implies$
$$\int_{[a,b]}\limninf\underline{s}(P_n)\dif m=\inf_{[a,b]}\limninf\overline{s}(P_n)\dif m,\mbox{ from }\limninf\underline{s}(P_n)\le\limninf\overline{s}(P_n)\mbox{, we have }\limninf\underline{s}(P_n)=\limninf\overline{s}(P_n)\alev$$
$$\implies\limninf\underline{s}(P_n)=f=\limninf\overline{s}(P_n)\alev\xRightarrow[(0)]{}f\mbox{ is continuous a.e.}$$
$f$ is continuous a.e.$\xRightarrow[(0)]{}\limninf\underline{s}(P_n)=f=\limninf\overline{s}(P_n)\alev$
$$\implies\int_{[a,b]}\limninf\underline{s}(P_n)\dif m=\int_{[a,b]}f\dif m=\int_{[a,b]}\limninf\overline{s}(P_n)\dif m=\int_a^bf\dif x$$
\textbf{Proposition:} $\exists\dint_{-\infty}^{+\infty}f\dif x:=\dlim_{\substack{a\to-\infty\\b\to+\infty}}\int_a^bf\dif x\implies\forall[a,b]\subset\R:f\mbox{ is continuous}\alev$\quad(clearly)\\
\textbf{Theorem:} $f\geq0\implies\dint_{-\infty}^{+\infty}f\dif x=\dint_\R f\dif m$\\
\textbf{Proof:} $\dint_{[-n,n]}f\dif m=\dint_{-n}^nf\dif x\incto\dint_{-\infty}^{+\infty}f\dif x\mbox{ is finite}\implies\limninf\int_{[-n,n]}f\dif m=\begin{cases}
    \dint_{-\infty}^{+\infty}f\dif x\\[8pt]
    \dint_\R f\dif m
\end{cases}$\\
\textbf{Remark:} The other type of improper integrals can be dealt with in a similar fashion\null\hfill{Content missing}
\subsection{Approximation of Measurable Functions}
\textbf{Theorem:} $\forall$ bounded measurable $f:[a,b]\to\R,\forall\varepsilon>0,\exists$ step function $h\st\dint_a^b|f-h|\dif m\le\varepsilon$\\
\textbf{Proof:}\\
Assume $f\geq0$, then 
$$\dint_{[a,b]}f\dif m=\sup\{\int_{[a,b]}\varphi\dif m\mid0\le\varphi\le f\mbox{, simple}\}\implies\exists\varphi\st\dint_{[a,b]}|f-\varphi|\dif m=\dint_{[a,b]}f\dif m-\dint_{[a,b]}\varphi\dif m\le\frac{\varepsilon}{2}$$
Suppose $\varphi:[a,b]\to\{a_1,\dots,a_n\}$, then
$$[a,b]=\cupin\varphi^{-1}(\{a_i\}),M:=\sup\{\varphi(x)\mid x\in[a,b]\}\implies\exists O_i\mbox{ open}\st E_i\subset O_i\land m(O_i)-m(E_i)\le\frac{\varepsilon}{2nM}$$
$$O_i=\cupjinf I_{ij}\mbox{ disjoint}\implies m(\cupjk I_{ij})=\sumjk m(I_{ij})\incto\sumjinf m(I_{ij})=m(O_i)\implies\exists k_i\st m(O_i)-m(\bigcup_j^{k_i}I_{ij})\le\frac{\varepsilon}{2nM}$$
$$G_i:=\cupjk I_{ij}\implies\int_{[a,b]}|\1_{E_i}-\1_{G_i}|\dif m\le m(E_i\Delta G_i)<\frac{\varepsilon}{nM}\quad(\mbox{set algebra skipped})$$
$$h:=\sumin a_i\1_{G_i}\implies\int_{[a,b]}|f-h|\dif m\le\int_{[a,b]}|f-\varphi|\dif m+\int_{[a,b]}|\varphi-h|\dif m\le\frac{\varphi}{2}+\frac{\varphi}{2}=\varepsilon\quad(\mbox{arithmetic skipped})$$
General case: $f^+,f^-$ can each be approximated arbitrarily closely by $h_1,h_2\implies$
$$\dint_{[a,b]}|f-(h_1-h_2)|\dif m\le\dint_{[a,b]}|f^+-h_1|\dif m+\dint_{[a,b]}|f^--h_2|\dif m<\frac{\varepsilon}{2}+\frac{\varepsilon}{2}=\varepsilon$$
\textbf{Theorem:} $\forall f\in\mathcal{L}^1,\forall\varepsilon>0:\exists$ continuous $g\st\dint|f-g|\dif m\le\varepsilon$\\
\textbf{Proof:}\\
Suppose $f$ is under the conditions of the last theorem, then
$$n:=k_i,\mbox{ re-order }\{I_{ij}\}_{j\le k_i}\mbox{ into }\{J_n\}_{m\le n},J_m=(c_m,d_m)\implies h=\summn a_m\1_{J_m},K:=\max_{m\le n}\{|a_n|\}$$
$$\varepsilon':=\min\left\{\frac{\varepsilon}{2nK},\min\{d_m-c_m\}\right\},g_m:=\begin{cases}
    1&x\in(c_m+\dfrac{\varepsilon'}{4},d_m-\dfrac{\varepsilon'}{4})\\
    0&x\notin J_m\\
    \mbox{linear interpolation}&\mbox{otherwise}
\end{cases}\mbox{ clearly, }g_m\mbox{ is continuous}$$
$$g:=\summn a_mg_m\implies\int_{[a,b]}|h-g|\dif m\le\summn a_m\sum_{[a,b]}|1_{J_m}-g_m|\dif m<nK\cdot\frac{\varepsilon'}{2}<\frac{\varepsilon}{2}$$
$$\implies\int_{[a,b]}|f-g|\dif m\le\int_{[a,b]}|f-h|\dif m+\int_{[a,b]}|h-g|\dif m<\varepsilon$$
For unbounded $f\in\mathcal{L}^1([a,b])$, suppose it is non-negative, then
$$f_n:=\min\{f,n\}\implies f_n\incto f\implies\forall\varepsilon>0,\exists N\st\int_{[a,b]}|f-f_N|\dif m\le\frac{\varepsilon}{2}$$
$$\implies\exists\mbox{ continuus }g\st\int_{[a,b]}|f_N-g|\dif m<\frac{\varepsilon}{2}\implies\int_{[a,b]}|f-g|\dif m<\varepsilon\quad(\mbox{extends trivially to }\mathcal{L}^1([a,b]))$$
$$\forall f\in\mathcal{L}^1\st f\geq0,\exists N\st\forall n>N:\int_{\{|x|\geq n\}}f\dif m\le\frac{\varepsilon}{3},\exists\mbox{ continuous }\int_{\{|x|\geq n\}}g\dif m<\frac{\varepsilon}{3}\st\int_{[-n,n]}|f-g|\dif m\le\frac{\varepsilon}{3}$$
$$\implies\int_\R|f-g|\dif m\le\int_{[-n,n]}|f-g|\dif m+\int_{\{|x|\geq n\}}f\dif m+\int_{\{|x|\geq n\}}g\dif m<\varepsilon\quad(\mbox{extends trivially to }\mathcal{L}^1(\R))$$
\textbf{Riemann-Lebesgue Lemma:}
$$f\in\mathcal{L}^1(\R),s_k:=\dint_\R f(x)\sin kx\dif x,c_k:=\int_\R f(x)\cos kx\dif x\implies\limkinf s_k=\limkinf c_k=0$$
\textbf{Proof:}
$$s_k=\int_\R f(x+\frac{\pi}{k})\sin(kx+\pi)\dif x=-\int_\R f(x+\frac{\pi}{k})\sin(kx)\dif x\implies$$
$$\int_\R|f(x)-f(x+\frac{\pi}{k})|\dif x\geq\left|\int_\R(f(x)-f(x+\frac{\pi}{k}))\sin(kx)\dif x\right|=2|s_k|$$
$\exists$ continuous $g\st g(x)=0$ when $x\notin[a,b]$ and $\dint_\R|f-g|\dif m<\dfrac{\varepsilon}{3}$, for $\dfrac{\pi}{k}<1:g(x+\dfrac{\pi}{k})=0$ when $x\notin[a-1,b+1]$
$$\implies|s_k|<\int_\R|f(x+\frac{\pi}{k})-f(x)|\dif m\le\int_\R|f(x+\frac{\pi}{k})-g(x+\frac{\pi}{k})|\dif m+\int_\R|g(x+\frac{\pi}{k})-g(x)|\dif m+\int_\R|g(x)-f(x)|\dif m<\varepsilon$$
(for sufficiently large $k$)\null\hfill{$\square$}
\clearpage
\section{Spaces of Integrable Functions}
\textbf{Note:} The spaces we will be working with are all infinite dimensional. Simply consider $\mathcal{C}([a,b],\R)$\\
A finite order polynomial can not be written as a linear combination of polynomials of lower order. Basis of $\mathcal{P}([a,b],\R):1,x,\dots,x^2$, hence it is infinite dimensional. Thus, $\mathcal{P}([a,b],\R)\subset\mathcal{C}([a,b],\R)\subset\mathcal{L}^1([a,b],\R)$
We shall also extend the integral to the complex space by writing $f=u+iv$, naturally, $\dint_Ef\dif m=\dint_Eu\dif m+i\dint_Ev\dif m$
\subsection{The Space $L^1$}
\textbf{Definition: Metric}
Let $X$ be a set, then any $d:X^2\to\R$ is a metric on $X$ and $(X,d)$ is a \textit{\textbf{metric space}}$\iff$
\begin{equation}
\begin{split}
    \mbox{\textbf{Non-negativity}}\,&\forall x,y\in X:d(x,y)\geq0\\
    \mbox{\textbf{Identical-0}}\,&d(x,y)=0\iff x=y\\
    \mbox{\textbf{Commutativity}}\,&\forall x,y\in X:d(x,y)=d(y,x)\\
    \mbox{\textbf{Triangle-inequality}}\,&d(x,z)\le d(x,y)+d(y,z)
\end{split}
\end{equation}
\textbf{Definition: Norm}
On vector space $X$ over $\R(\C),f:X\to||x||$ is a norm$\iff$
\begin{equation}
\begin{split}
    \mbox{\textbf{Non-negativity}}&\,\forall x\in X:||x||\geq0\\
    \mbox{\textbf{Identical-0}}&\,x=0\iff||x||=0\\
    \mbox{\textbf{Absolute multiplicativity}}&\,||\alpha\cdot x||=|\alpha|\cdot||x||\\
    \mbox{\textbf{Triangular Inequality}}&\,||x+y||\le||x||+||y||
\end{split}
\end{equation}
$||\cdot||:x\mapsto||x||$ defines a metric by $d(x,y)=||x-y||$, the triangle inequality follows from
$$||x-z||=||(x-y)+(y-z)||\le||x-y||+||y-z||$$
To define a norm on $\mathcal{L}^1(E)$ with $E\subset\mathcal{M}(\R)$, we have to establish an equivalence class between functions which are equal a.e. to avoid non-identical 0s.\\
\textbf{Definitions:} $$L^1(E):=\mathcal{L}^1(E)/_\equiv,f\equiv g:=f=g\alev,[f]:=\{f^*\in\mathcal{L}^1(E)\mid f^*=f\alev\}$$
$$[f]+[g]:=[f+g],\forall a\in\R:a[f]:=[a\cdot f]\implies L^1(E)\mbox{ is a vector space}\quad\forall[f]\in L^1(E),||[f]||_1:=\int_E|f|\dif m$$
This norm is well defined, as we have shown in previous chapters.\\
From here on, we use $f$ and $[f]$ interchangeably to not obscure the underlying ideas with notation.\\
\textbf{Definition: Cauchy Sequence}\\
Let $X$ be a vector space with norm $||\cdot||_X,f_n\in X$ is cauchy if $\forall\varepsilon>0,\exists N\st\forall n>N:||f_n-f_m||_X<\varepsilon$\\
If all Cauchy sequence in $X$ converge to an element of $X$, then $X$ is \textit{\textbf{complete}}.\\
\textbf{Example:} $f_n:=\dfrac{1}{x^2}\1_{(0,n)}$
$$||f_n-f_m||_1=\int_0^\infty\frac{1}{x^2}|\1_{(0,n)}-\1_{(0,m)}|\dif x\,(\mbox{WLoG, suppose }n\geq m)=\int_m^n\frac{1}{x^2}\dif x=\frac{1}{m}-\frac{1}{n}\to0\implies\{f_n\}\mbox{ is Cauchy}$$
\textbf{Proposition:} The absolute convergence theorem is equivalent to the completeness theorems of $\R$\\
We know that the Cauchy convergence theorem implies ACT, now we prove the other direction.\\
\textbf{Proof:} Suppose $\{x_n\}\subset\R$ is Cauchy, then $\exists\{x_{n_k}\}\subset\{x_n\}\st\forall n\geq n_k:|x_n-x_{n_k}|\le2^{-k}$
$$|x_{n_1}|+\sumik|x_{n_i}-x_{n_{i-1}}|<|x_{n_1}|+\sumik2^{-i}\to|x_{n_1}|+1\implies$$
$$|x_{n_1}|+\sumik|x_{n_i}-x_{n_{i-1}}|\mbox{ converges}\xRightarrow{\mbox{ACT}}x_{n_k}=x_{n_i}+\sumik(x_{n_i}-x_{n_{i-1}})\mbox{ converges}$$
\textbf{Theorem:} $L^1(E)$ is complete\\
\textbf{Proof:}\\
Suppose $f_n$ is Cauchy, then $\forall k,\exists N_k\st\forall n\geq N_k:||f_n-f_{N_k}||\le2^{-k}\implies$\\
$\sumninf||f_{N_{k+1}}-f_{N_1}||_1$ is finite $\xRightarrow{\mbox{BLT}}\sumninf|f_{N_{n+1}}-f_{N_n}|$ converges a.e. $\xRightarrow{\mbox{ACT}}\sumninf(f_{N_{n+1}}-f_{N_n})$ converges a.e.$\implies$\\
$f_{N_{k+1}}=f_{N_1}+\sumnk(f_{N_{n+1}}-f_{N_n})$ converges a.e., suppose it converges to $f$, then trivially, $f_n\to f$ point-wise a.e.
$$\mbox{Now we prove convergence under }||\cdot||_1\mbox{ and }f\in L^1:\quad\forall\varepsilon>0,\exists N\st\forall m,n\geq N:||f_n-f_m||_1\le\varepsilon\implies$$
$$||f-f_m||_1=\int|f-f_m|\dif m\le\limkinf\inf\int|f_{N_k}-f_m|\dif m=\limkinf\inf||f_{N_k}-f_m||_1<\varepsilon\implies$$
$$||f-f_m||_1\to0\implies f-f_m\in[0]\subset L^1\implies f=(f-f_m)+f_m\in L^1\quad\null\square$$
\subsection{The Hilbert Space $L^2$}
\textbf{Definition:}
$$||f||_2:=(\int_E|f|^2\dif m)^\frac{1}{2},\mathcal{L}^2(E):=\{f\in E\mbox{ is measurable}:||f||_2<\infty\}$$
\null\hfill{Note: the modulus around $f$ keeps the integral real on complex spaces}\\
\textbf{Definition:} $L^2(E):=\mathcal{L}^2(E)/_\equiv$ (same $\equiv$ as $L^1$)\\
\textbf{Corollary:} $L^2(E)$ is a vector space\\
\textbf{Proof:}
To prove closure under addition, note $|f+g|^2\le2^2\max\{|f|^2,|g|^2\}\le4(|f|^2+|g|^2)\quad\square$
\subsubsection{Properties of the $L^2$-norm}
Non-negativity, identical-0 and absolute multiplicativity are immediate, now we prove the triangle inequality\\
\textbf{Corollary:} $|\dint_\C f\dif m|\le\dint_\C|f|\dif m$\\
\textbf{Proof:}
$$|\int_\C f\dif m|=|\int_\C f_1+if_2\dif m|=|\int_\R f_1\dif m+i\int_\R f_2\dif m|=||\int_\R(f_1,f_2)\dif m||_\R\implies$$
$$||\int_\R(f_1,f_2)\dif m||_\R^2=\int_\R\langle(f_1,f_2),\int_\R(f_1,f_2)\dif m\rangle\dif m\le\int_\R||(f_1,f_2)||_\R\cdot||\int_\R(f_1,f_2)\dif m||_\R\dif m$$
$$=\int_\R||(f_1,f_2)||_\R\dif m\cdot||\int_\R(f_1,f_2)\dif m||_\R\implies||\int_\R(f_1,f_2)\dif m||_\R\le\int_\R||(f_1,f_2)||_\R\dif m\iff|\int_\C f\dif m|\le\int_\C|f|\dif m$$
\textbf{Schwarz Inequality:} If $f,g\in L^2(E,\C)$, then $f\cdot g\in L^1(E,C)$ and $|\dint_Ef\cdot\overline{g}\dif m|\le||f\cdot g||_1\le||f||_2\cdot||g||_2$\\
\textbf{Proof:}
$$|\int_Ef\cdot\overline{g}\dif m|\le\int_E|f|\cdot|g|\dif m=||f\cdot g||_1$$
Now $f$ for $|f|$, $g$ for $|g|$, and to avoid infinity $L^2$ norms, define
$$f_n:=\min\{f,n\},g_n:=\{g,n\},E_k:=E\cap[-n,n]\implies$$
$$\forall t\in\R:0\le\int_{E_k}(f_n+g_n)^2\dif m=\int_{E_k}f_n^2\dif m+2t\int_{E_k}f_ng_n\dif m+t^2\int_{E_k}g_n^2\dif m$$
The above quadratic of $t$ is non-negative, hence can not have two distinct solutions. Taking the determinant:
$$(\int_{E_k}f_n\cdot g_n\dif m)^2\le\int_{E_k}f_n^2\dif m\cdot\int_{E_k}g_n^2\dif m\le\int_{E_k}|f|^2\dif m\cdot\int_{E_k}|g|^2\dif m=||f||_2^2\cdot||g||_2^2\xRightarrow{\mbox{MCT}}(\int_Ef\cdot g\dif m)^2\le||f||_2^2\cdot||g||_2^2$$
\textbf{Triangle Inequality} for $L^2(E,\C)$\\
\textbf{Proof:}
\begin{equation}
\begin{split}
    ||f+g||_2^2&=\int_E|f+g|^2\dif m=\int_E(f+g)\overline{(f+g)}\dif m\\
    &=\int_E|f|^2\dif m+\int_Ef\cdot\overline{g}+\overline{f}\cdot g\dif m+\int_E|g|^2\dif m\le||f||_2^2+2||f||_2\cdot||g||_2+||g||_2^2=(||f||_2+||g||_2)^2
\end{split}
\end{equation}
A more general proof for the completeness of $||\cdot||_p$ norm will be provided in the next subsection.\\
\textbf{Note:} $L^1\not\subset L^2\land L^2\not\subset L^1$\\
\textbf{Example:}
\begin{equation}
\begin{split}
    E=[1,\infty),f=\frac{1}{x}\implies f\in L^2(E)\land f\notin L^1(E)\\
    F=(0,1),g=\frac{1}{\sqrt{x}}\implies g\in L^1(E)\land f\notin L^2(E)
\end{split}
\end{equation}
\textbf{Proposition:} $m(D)\le\infty\implies L^2(D)\subset L^1(D)$\\
\textbf{Proof:} $f\in L^2(E)\implies\dint_D|f|^2\dif m$ is finite$\implies|f|$ is finite a.e.$\implies\dint_D|f|\dif m$ is finite$\implies f\in L^1(D)$
\subsubsection{Inner Product Spaces}
\textbf{Definitions:} $\forall f,g\in L^2(E,\C),\langle f,g\rangle:=\dint f\cdot\overline{g}\dif m\implies\sqrt{\langle f,f\rangle}=(\dint_Ef\cdot\overline{f}\dif m)^\frac{1}{2}=(\dint_E|f|^2\dif m)^\frac{1}{2}=||f||_2$\\
The conjugate linearity, conjugate symmetry, and positive definiteness $(\langle f,f\rangle)\geq0$ are trivially true. Hence, $L^2$ is a inner product space with respect to the inner-product generated by its norm.\\
The following are hence immediate:\\
\textbf{Identities:}\\
\begin{equation}
\begin{split}
    \mbox{\textbf{Parallelogram Law:}}\,&||h_1+h_2||^2+||h_1-h_2||^2=2(||h_1||^2+||h_2||^2)\\
    \mbox{\textbf{Polarisation Identity:}}\,&\langle h_1,h_2\rangle=\frac{1}{4}\cdot\left(||h_1+h_2||^2-||h_1-h_2||^2+i\cdot\left(||h_1+ih_2||^2-||h_1-ih_2||^2\right)\right)
\end{split}
\end{equation}
\textbf{Definition:}
\quad\textbf{Banach Spaces:} vector spaces complete under their corresponding norms\\
\quad\textbf{Hilbert Spaces:} Banach spaces whose norms are induced by their inner product\\
\textbf{Corollary:} (proof not provided)\\
Hilbert spaces are precisely the spaces where the parallelogram law holds, the inner-product is then recovered from the norm via the polarisation identity.\\
\textbf{Example:} $L^1([0,1])$ is not a Hilbert space. To see this, put $h_1=x-\frac{1}{2},h_2=\frac{1}{2}$ through the parallelogram law\\
\subsubsection{Orthogonality and Projections}
\textbf{Definition:} $\cos\theta=\dfrac{\langle g,h\rangle}{||g||\cdot||h||}$\\
\textbf{Theorem:} Let $K$ be a complete subspace of the Hilbert space $H$, then $\forall h\in H,\exists! h'\in K\st\forall k\in K:\langle h-h',k\rangle=0$. Or equivalently: $||h-h'||=\inf\{||h-k||:k\in K\}$\\
\textbf{Proof:} We begin with proving the equivalence between the two conditions
$$\forall k\in K:\langle h-h',k\rangle=0\implies||h-k||^2=||(h-h')+(h'-k)||^2=||h-h'||^2+||h'-k||^2\geq||h-h'||^2$$
$$||h-h'||=\inf\{||h-k||\}\implies\forall t\in\R,\forall k\in K:||h-(h'+tk)||^2\geq||h-h'||^2\implies t^2||k||^2-2t\mbox{Re}\langle h-h',k\rangle\geq0$$
After basic quadratic manipulation, we have: 
$$\mbox{Re}\langle h-h',k\rangle=0\implies\forall k\in K:0=\mbox{Re}\langle h-h',-ik\rangle=\mbox{Im}\langle h-h',k\rangle\implies\langle h-h',k\rangle=0$$
We now prove the existence of $h'$\\
Choose $\{k_n\}\subset K\st||h-k_n||\to\inf\{||h-k||:k\in K\}$, apply the parallelogram law:
$$||h-k_n||^2+||h-k_m||^2=2\left(||h-\frac{1}{2}(k_m+k_n)||^2+||\frac{1}{2}(k_m-k_n)||^2\right)$$
LHS $\to2(\inf\{||h-k||\})^2$, RHS $\geq2(\inf\{||h-k||\})^2\implies||k_m-k_n||\to0$, hence $\{k_n\}$ is Cauchy, converges to some $h'$
$$\inf\{||h-k||\}\le||h-h'||\le||h-k_n||+||k_n-h'||\to\inf\{||h-k||\}\implies||h-h'||=\inf\{||h-k||\}$$
\subsection{The $L^p$ Spaces: Completeness}
\textbf{Definitions:} $\forall p\in\N^{\geq1},L^p:=\{f:\dint_E|f|^p\dif m<\infty\},||f||_p:=(\dint_E|f|^p\dif m)^\frac{1}{p}$\\
$f$ is \textit{\textbf{Essentially bounded}} if $\esssup|f|\le\infty,L^\infty(E):=\{f:f$ is essentially bounded$\},||f||_\infty:=\esssup f$\\
Clearly, $\forall p,L^p(E)$ is a vector space, most norm conditions also stand trivially. The non-trivial properties are the triangle inequality completeness under the corresponding norm.\\
\textbf{Triangle Inequality}\\
\textbf{Proof:} For $L^\infty:|f+g|\le|f|+|g|\implies||f+g||_\infty\le||f||_\infty+||g||_\infty$, now we look at $L^p$:\\
\textbf{Hölder's Inequality:} 
\begin{equation}
\begin{split}
    (1)\,&\forall x,y\in\R^{\geq0},\forall\alpha,\beta\in(0,1)\st\alpha+\beta=1:x^\alpha y^\beta\le\alpha x+\beta y\\
    (2)\,&\forall p,q\in\R^{>1}\st\frac{1}{p}+\frac{1}{q}=1:\int|f\cdot\overline{g}|\dif m\le(\int|f|^p\dif m)^\frac{1}{p}\cdot(\int|g|^q\dif m)^\frac{1}{q}\iff||f\cdot g||_1\le||f||_p\cdot||g||_q
\end{split}
\end{equation}
We have proved this inequality previously in ``Classical Real Analysis", now we provide a different proof\\
\textbf{Proof(1):} When $x=0$, the claim is obvious, we take $x>0$:
$$f(t):=(1-\beta)+\beta t-t^\beta\,(t\in[0,\infty]),f'(t)=\beta(1-t^{\beta-1})\begin{cases}
    <0&t\in(0,1)\\
    >0&t\in[0,\infty)
\end{cases}$$
Hence $f(1)=0$ is the only minimum on $[0,\infty)\implies f(t)\geq0$ Let $t$ be $\dfrac{x}{y}$ then trivially\quad$\square$\\
\textbf{Proof(2):} Assume that $||f||_p=||g||_q=1$, this can be done by dividing on both sides. Hence, we only have to show that $||f,g||_1\le1$
$$|f\cdot\overline{g}|=(|f|^p)^\frac{1}{p}(|g|^q)^\frac{1}{q}\le\frac{1}{p}|f|^p+\frac{1}{q}|g|^q\implies\int|f\cdot\overline{g}|\dif m\le\frac{1}{p}\int|f|^p\dif m+\frac{1}{q}\int|g|^q\dif m=\frac{1}{p}+\frac{1}{q}=1$$
\textbf{Note:} $L^\infty$ is not a Hilbert space, take $h_1=0,h_2=x-1$ for the parallelogram law.\\
\textbf{Minkowski's Inequality:} $\forall p\geq1,\forall f,g\in L^p(E):||f+g||_p\le||f||_p+||g||_p$\\
\textbf{Proof:}\\
Suppose $1<p<\infty$, we have: $|f+g|^p\le|f|\cdot|f+g|^{p-1}+|g|\cdot|f+g|^{p-1}$\\
Take $q\st\dfrac{1}{p}+\dfrac{1}{q}=1$, there is $|f+g|^{(p-1)q}=|f+g|^p\le\infty\implies(f+g)^{p-1}\in L^q$\\
$$\implies(\int|f+g|^p\dif m)^{\frac{1}{p}+\frac{1}{q}}=\int|f+g|^p\dif m\le\int|f|\cdot|f+g|^{p-1}\dif m+\int|g|\cdot|f+g|^{p-1}\dif m$$
$$\le(\int|f|^p\dif m)^\frac{1}{p}(\int|f+g|^{(p-1)q}\dif m)^\frac{1}{q}+(\int|g|\dif m)^\frac{1}{p}(\int|f+g|^{(p-1)q}\dif m)^\frac{1}{q}$$
$$=(\int|f+g|^p\dif m)^\frac{1}{q}\left((\int|f|^p\dif m)^\frac{1}{p}+(\int|g|^p\dif m)^\frac{1}{p}\right)\implies(\int|f+g|^p\dif m)^\frac{1}{p}\le(\int|f|^p\dif m)^\frac{1}{p}+(\int|g|^p\dif m)^\frac{1}{p}\quad\square$$
\textbf{Completeness Theorem of the $L^p$ Norm}\\
\textbf{Proof:} Suppose $f_n$ is Cauchy, then $\exists n_k\st\forall n\geq n_k:||f_n-f_{n_k}||v$
$$\implies\int(\sumiinf|f_{n_{i+1}}-f_{n_i}|)^p\dif m=\int\limkinf(\sumik|f_{n_{i+1}}-f_{n_i}|)^p\dif m\le\limkinf\inf\int(\sumik|f_{n_{i+1}}-f_{n_i})^p\dif m$$
$$=\left|\left|\sumik|f_{n_{i+1}}-f_{n_i}|\right|\right|_p^p\le(\sumik||f_{n_{i+1}}-f_{n_i}||_p)^p<1\implies\sumiinf|f_{n_{i+1}}-f_{n_i}|\mbox{ is finite a.e.}$$
Note: We are using Fatou instead of BLT as the summation operation can not be extracted from the $p$th power
$$\implies f_{n_k}=f_{n_1}+\sumik(f_{n_{i+1}}-f_{n_i})\to f\alev$$
$$||f-f_m||_p^p=\int|f-f_m|^p\dif m\le\limiinf\inf\int|f_{n_i}-f_m|^p\dif m\le\varepsilon^p\implies f-f_m\in L^p\implies f=f_m+(f-f_m)\in L^p\quad\square$$
\textbf{Theorem:} $m(E)<\infty\implies\forall 1\le p\le q\le\infty:L^p(E)\subset L^p(E)$\\
\textbf{Proof:}
$$|f(x)|^p\le1+|f(x)|^q\implies\int_E|f|^p\dif m\le\int_E1\dif m+\int_E|f|^q\dif m=m(E)+\int_E|f|^q\dif m$$
Hence if $\dint_E|f|^q\dif m$ is finite, then $\dint_E|f|^p\dif m$ is finite\quad$\square$
\clearpage
\section{Product Measures}
\subsection{Multi-dimensional Lebesgue Measure}
\subsection{Product $\sigma$-fields}
\subsection{Construction of the Product Measure}
\subsection{Fubini's Theorem}
\clearpage
\section{The Radon-Nikodym Theorem}
\subsection{Densities and Conditioning}
\subsection{Lebesgue-Stieltjes Measures}
\subsubsection{Construction of Lebesgue-Stieltjes Measures}
\subsubsection{Absolute Continuity of Functions}
\subsubsection{Functions of Bounded Variation}
\subsubsection{Hahn-Jordan Decomposition}
\subsubsection{Signed Measures}
\clearpage
\section{Limit Theorems}
\subsection{Modes of Convergence}
\clearpage
\end{document}/